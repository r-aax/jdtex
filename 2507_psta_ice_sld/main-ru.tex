\documentclass{psta}% Подгружает макропакеты:
% \usepackage{inputenc,fontenc,babel,hyperref,url,xcolor,textcase,backref,enumitem,
%       amssymb,amscd,euscript,nameref,graphics,pdflscape,upref,rotating,array}
% Авторы могут использовать другие общедоступные макропакеты.
\usepackage{soulutf8}
\usepackage{fancyvrb}
\usepackage{attachfile2}
\usepackage{makecell}
\newcommand\thinhline{\Xhline{0.1pt}}
\newcommand\thinvline{\textcolor{lightgray}{\vrule width .2pt}}
%\usepackage{ragged2e}
%\tracingall
%\tracingnone

\selectlanguage{russian} % Не забывайте отметить возврат на русский язык
\newcommand\vitem[1][]{\SaveVerb[%
    aftersave={\item[\footnotesize\em{\UseVerb[#1]{vsave}}]}]{vsave}}
% Пожалуйста, уточните классификация Вашей статьи согласно УДК и MSC
\subjclass[UDC]{004.915+808.2:347.777}
\subjclass[2020]{97P99; 97U99}
% Выберите и укажите, пожалуйста, тип статьи и номер рубрики, см.\ ниже
%\ArticleType {RAR} % Research Article
\JournalRubric {0} % см. пояснение в тексте
%
\title [Шаблон стилевого файла]{Готовим  в~\LaTeXe~статью в~журнал «Программные~системы: теория~и~приложения»\subtitle{ш}аблон стилевого файла}
% \title [Сокращённое название для колонтитула]{Полный заголовок статьи}
\thanks{Работа поддержана грантом XXX №  YYY-ZZZ}


% Аналогично для каждого из остальных авторов
\author{Знаменский, Сергей Витальевич}
\CRediT{1,2,3[создание стилевого файла],7,9,10,11,13}{99.95\%}
\email{svz@latex.pereslavl.ru} % Электронный адрес автора
\correspondent % отмечен автор, участвующий в переписке
\address{\PSI} % Для организаций авторов, публиковавших статьи в журнале, можно использовать макроопределения из psta/psta-utf8.sty
\info{д.ф.-м.н., гл. научн. сотр. Института программных систем имени А.\,К.~Айламазяна РАН. Научные интересы мигрируют от функционального анализа и аналогов выпуклости в~$\mathbb C^n$ к~прозрачным и действенным алгоритмам}
\imageauthor{Foto by A. Yu. Fomenko, CC-BY-SA}
\orcid{0000-0001-8845-7627}
\image{pics/svz}

% Фамилия, запятая, пробел, имя отчество автора
\author{Фамилия соавтора, Имя Отчество}
\address{\href{https://ru.wikipedia.org/wiki/\%D0\%A3\%D0\%BD\%D0\%B8\%D0\%B2\%D0\%B5\%D1\%80\%D1\%81\%D0\%B8\%D1\%82\%D0\%B5\%D1\%82}{Университет соавтора, город, страна}}
\address{\PSI} % Для организаций авторов, публиковавших статьи в журнале, можно использовать макроопределения из psta/psta-utf8.sty
\email{first.author@mail.ru}
\info{здесь в свободной форме в 5-9 строках описывается официальный статус соавтора, его научных интересы и достижения}
\image{pics/nobody}% файл фотографии автора
\orcid{0000.0000.0000.0000}
\CRediT{10}{0.05\%} % Участие автора, см. пояснение в тексте

\begin{abstract}
Новый дизайн статьи журнала «Программные системы: теория и приложения» включает недавно ставшую стандартной информацию.
Элементы оформления титульной информации научной статьи формализованы в соответствии с ГОСТ Р 7.0.7-2021.
Приведенные примеры оформления основного текста также включают такие сложные ситуации, как заголовок с подзаголовком
и таблицу на полную повёрнутую страницу.

Качественное авторское оформление в \LaTeXe\ ускоряет публикацию в журнале.
Шаблон адресуется авторам, оформляющим свою рукопись в \LaTeXe, и содержит заготовку с примерами и пояснениями к использованию важнейших макросов стилевого файла для подготовки статьи и \Href{http://psta.psiras.ru/publ-conditions/publ-conditions.html}{требования редакции}.

\end{abstract}
\keywords{Стилевой файл, \LaTeXe, дизайн издания, двуязычная публикация}

% Все метаданные должны также присутствовать на английском языке,
% заключённые в  \selectlanguage{english}...,\selectlanguage{russian}:
\selectlanguage{english}
\title[Template for submission]{Prepare article in \LaTeXe\subtitle (template for submission to the journal)}
% \title [Short title for top colontitle]{The Full Engish Title}
% Last name, coma other names
\thanks{The work is supported by the grant XXX №  YYY-ZZZ}

\author{Znamenskij, Sergej Vital'evich}
\address{\PSI}
\info{Research interests migrated from research in Functional Analysis, Complex Analysis and finite-dimensional Projective Geometry (analogues of Convexity) to  the Research of Practical Computational Algorithms}
\email{svz@latex.pereslavl.ru}
\orcid{0000-0001-8845-7627}
\authorid{1507}

\author{Family Name, Other Names}
\address{University, town, country}
\address{\PSI}
% Other information about author should present either in English

\begin{abstract}
The new design of the article in the journal "Program systems: theory and applications" includes information that has recently become standard.
Elements of design of the title information of a scientific article are formalized in accordance with GOST R 7.0.7-2021.
The given examples of body text design also include such complex situations as a heading with a subheading
and a full-page table.

High-quality author's design in \LaTeXe\ speeds up publication in the journal.
The template is addressed to authors preparing their manuscript in \LaTeXe, and contains a template with examples and explanations for using the most important style file macros for preparing an article and \Href{http://psta.psiras.ru/publ-conditions/publ-conditions.html }{editorial requirements}.
\end{abstract}
\keywords{here are comme delimited words and combinations of words to search, a couple of lines}
\selectlanguage{russian} % Не забывайте отметить возврат на русский язык
%

\begin{document}
\Russian
\maketitle
% Если список авторов не умещается в колонтитул, используйте \shortauthors{...}

\section*{Введение}

\Href{http://psta.psiras.ru/publ-conditions/psta.zip}{Стилевой файл} журнала \Href{http://psta.psiras.ru/}{«Программные системы: теория и приложения»}  дополняет стандартный \LaTeXe\ макрокомандами, упрощающими высококачественную подготовку двуязычных изданий и облегчающими совместную проработку русскоязычной и англоязычной версии.

Оформление русскоязычной версии ориентировано на закреплённые в российских ГОСТах традиции, а оформление англоязычной\---на современное состояние классических традиций американского математического общества.

Текст шаблона содержит примеры использование макросов стилевого файла с краткими пояснениями и рекомендациями.
\Altref{Раздел}{sec:general} характеризует общие требования к использованию пакета \verb|psta|.
\Altref{В~разделе}{sec:meta} описываются макрокоманды для оформления научной статьи согласно ГОСТ Р 7.0.7-2021.
\Altref{Раздел}{sec:text} знакомит со стилем журнала и особенностями оформления элементов текста, иллюстраций, гиперссылок, библиографии.

\section{Назначение макропакета и подготовка к его использованию}\label{sec:general}
Макропакет доступен \Href{http://psta.psiras.ru/publ-conditions/psta.zip}{по ссылке}, создан для обработки современным \texttt{pdflatex}, но допускает и другие способы обработки LaTeX.

\subsection{Подготовка к работе}
Ожидается, что авторы знакомы с общепринятыми требованиями
\begin{itemize}
\item к структуре и содержанию научных публикаций \cites{Safonov2007,Fradkov2003,Sviderskaya2011,KirillovaLong,KirillovaShort},
\item к набору в \LaTeX\ \cites{Stolarov,Vorontsov,Syutkin},
\item к оформлению сложных формул в  AMS\LaTeX\ \cites{AMSshort,Mathmode}.
\end{itemize}

Для улучшения качества pdf и получения возможности доводки корректуры полезно установить пакет \verb|cm-super|.

Перед началом работы рекомендуется раскрыть с~поддиректориями архив psta.zip %\attachfile{psta.zip}
и обработать файл \textsf{mail-ru.tex} в Вашей системе. К сожалению, LaTeX быстро меняется, и, например, базовая установка Миктех, если не включать обновления, застревает на символе номера.
Если компиляция \textsf{main-ru.pdf} прошла успешно, то можно поменять в \textsf{main-ru.tex} заполнение титульной информации, а затем и содержимое.



\subsection{Опции макропакета}
Несколько опций могут использоваться в первой строке исходного файла \verb|\documentclass[|...\verb|]{psta}| в квадратных скобках:
\begin{description}
% Команда \vitem из пакета fancyvrb заменяет неработающее сочетание \item \verb
\vitem |russain|, \verb |english| настраивают оформление для статей на русском языке или на английском языке;
\vitem |blind| для отправки PDF рецензенту аккуратно прячет персональную информацию. Если в тексте статьи есть фрагменты, которые могут идентифицировать автора, то автор прячет их в аргумент команды \verb|\HideFromPeer{|...\verb|}|, чтобы избежать лишнего редактирования;
\vitem |secsymbol| добавляет символ параграфа в заголовки, при этом  гиперссылка \verb|\ref{|метка\footnote{русские буквы в метках пока не работают.}\verb|}| на (под)раздел  \verb|\label{|метка\verb|}| приобретает желаемый вид как в \Href{http://psta.psiras.ru/read/psta2018_1_53-83.pdf\#page=7}{этой опубликованной статье}.
\vitem |sloppy| заставляет команду \verb|\sloppy| делать пробелы между словами и формулами более растяжимыми только до конца текущего абзаца.
\end{description}

\section{Титульная информация о статье}\label{sec:meta}
Преамбула исходного файла (до \verb|\begin|\verb|{document}|) должна содержать всю титульную информацию, включая информацию об авторах на русском и английском языках.
Язык переключает команда  \verb|\selectlanguage|\verb|{|...\verb|}|.

\subsection{Тип и рубрика статьи}

Отличный от RAR тип статьи указывается командой \verb|\ArticleType{|...\verb|}| с аргументом из \Altref{таблицы}{tab:ArticleType},
\begin{table}\footnotesize
\caption{Типы статей}\label{tab:ArticleType}
\begin{tabular}{|c|m{30mm}<{\raggedright}!{\thinvline}m{61mm}|}
\hline
Аргумент&\hfil\hfil Значение	 &\hfil Пояснение\\
\hline
PER	& Персоналия		 (Personal)& От редакционной коллегии.\\\thinhline
RAR	& Научная статья 	 (Research Article)& Текст статьи должен ясно описывать, обосновывать и обсуждать значимо новые уникальные результаты и не содержать 
повторений, ненужных для понимания результатов, их новизны и значимости. 
Численная оценка заимствований не значима.\\\thinhline
REV	& Обзорная статья	 (Review Article)&Обзор, по-новому представляющий состояние исследований в области и выявляющий ключевые направления дальнейшего развития.\\\thinhline
SCO	& Краткое сообщение 	 (Short~Communication)& По особому решению редакционной коллегии. \\
\hline
\end{tabular}
\end{table}
(как в метаданных \Href{https://www.elibrary.ru/}{E-library}).

Номер рубрики \verb|\JournalRubric{|...\verb|}| авторы выбирают на основании \Altref{таблицы}{tab:JournalRubric}, в которой зачёркнуты прежние названия рубрик.
\begin{table}\footnotesize\renewcommand{\arraystretch}{1.3}
\caption{Рубрики журнала и научные специальности}\label{tab:JournalRubric}
\begin{tabular}{|c|>{\raggedright}p{37mm}<{\footnotesize}!{\thinvline}>{\raggedright}p{41mm}<{\footnotesize}!{\thinvline}>{\raggedright}p{14mm}<{\footnotesize}|}
\hline
№ & Наименование рубрики\centering & Специальности ВАК\centering & Отрасли науки\centering\arraybackslash\\
\hline
1 & \st{Информационные системы в культуре и образовании} Прикладные программные системы & 2.3.5. Математическое и программное обеспечение вычислительных систем, комплексов и компьютерных сетей & физ.-мат., техн. \arraybackslash\\\thinhline
2 & Информационные системы в медицине & 3.3.9. Медицинская информатика & мед. \arraybackslash\\\thinhline
3 & \st{Информационные системы в экономике} Прикладные программные системы &2.3.5. Математическое и программное обеспечение вычислительных систем, комплексов и компьютерных сетей & физ.-мат., техн. \arraybackslash\\\thinhline
4 & Искусственный интеллект, интеллектуальные системы, нейронные сети &1.2.1. Искусственный интеллект и машинное обучение & физ.-мат.,  \arraybackslash\\\thinhline
5 & \st{Математические основы программирования} Теоретические основания программных систем & 2.3.7. Компьютерное моделирование и автоматизация проектирования & физ.-мат., техн. \arraybackslash\\\thinhline
6 & Математическое моделирование & 1.2.2. Математическое моделирование, численные методы и комплексы программ & техн. \arraybackslash\\\thinhline
7 & Методы оптимизации и теория управления & 2.3.1. Системный анализ, управление и обработка информации & физ.-мат., техн. \arraybackslash\\\thinhline
8 & \st{Программное и аппаратное обеспечение для суперЭВМ} Программное и аппаратное обеспечение распределенных и суперкомпьютерных систем & 2.3.5. Математическое и программное обеспечение вычислительных систем, комплексов и компьютерных сетей, \newline2.3.7. Компьютерное моделирование и автоматизация проектирования & физ.-мат., техн. \arraybackslash\\
\hline
\end{tabular}
\end{table}

\subsection{УДК, MSC, заголовок, аннотация}
Коды классификаторов статьи заполняются в командах

\noindent
\verb|\subjclass [UDC]{|\Href{https://teacode.com/online/udc/}{...}\verb|}| и  \verb|\subjclass [2020]{|\Href{https://mathscinet.ams.org/msnhtml/msc2020.pdf}{...}\verb|}|.

Заглавие статьи (не более 20 слов) оформляется \verb|\title{|...\verb|}|, но
для длинных заголовков в верхнем колонтитуле используется необязательный краткий вариант заголовка
\verb|\title [|...\verb|]{|...\verb|}|.
Если нужен подзаголовок, то команда  \verb|\subtitle| отделяет его в заголовке. В этом случае либо первая буква подзаголовка заключается в фигурные скобки, чтобы стать заглавной в библиографической ссылке, либо подзаголовок заключается в скобки.

Аннотация, оформляемая окружением \verb|abstract|, должна лаконично (100$\div$250 слов) раскрыть основные результаты статьи и их научную ценность.
Ключевые слова и фразы \verb|\keywords{|...\verb|}| (3$\div$15 слов и сочетаний через запятую) для улучшения поиска статьи.

\subsection{Макрокоманды для информации об авторе}

Команда {\verb|\author{|...\verb|}|} воспринимает фамилию автора (например, von Neumann) и его имя с отчеством либо список имён.
Согласно аргументации в руководстве к пакету \Href{http://www.ams.org/arc/tex/amsrefs/amsrdoc.pdf}{AMSRefs} и ГОСТ Р 7.0.100-2018, фамилия отделяется запятой. После запятой в русском языке ставится пробел. Особый признак при наличии также отделяется запятой с пробелом, например,

\verb|\author{Лаврентьев, {М}ихаил {М}ихайлович, мл.}|

\noindent
Первые буквы русских имён и отчества (как и нераздельные  сочетания начальных английских букв Yu, Sh и др.) выделяются фигурными скобками для надёжности автоматического формирования элементов оформления статьи.

Информация об авторе прописывается в аргументах команд:


\begin{description}
\vitem |\address{|\!\!\!...\verb|}| место работы(ссылка на сайт), город, страна;
\vitem |\thanks{|\!\!\!...\verb|}| благодарности или информация о~поддержке;
\vitem |\email{|\!\!\!...\verb|}| адрес электронной почты;
\vitem |\orcid{|\!\!\!...\verb|}| идентификатор авторской регистрации в \Href{www.orcid.org}{ORCID};
\vitem|\info{|\!\!\!...\verb|}| описание в 4-7 строчках официального статуса автора, его научных интересов или достижений;
\vitem|\image{|\!\!\!...\verb|}| имя файла с качественной фотографией лица автора.
\end{description}

Те из команд \verb|\thanks{|...\verb|}| и \verb|\address{|...\verb|}|, которые относятся ко всем авторам, предшествуют всем командам \verb|\author|,
остальные приводятся (возможно дублируются) после каждого \verb|\author|, к которому они относятся.
Это дублирование автоматически преобразуется в компактное представление связей на титульной странице статьи.
Автор, ответственный за переписку, выделяется командой \verb|\correspondent|.

\Altref{Таблица}{tab:CRediT} содержит общепринятые термины описания вклада
авторов, введенные в \cite{CRediT} и активно используемые ведущими издательствами
\linebreak \begin{landscape} % вставка широкой таблицы или иллюстрации между строками абзаца
\begin{table}\footnotesize
\caption{Онтология авторского вклада}\label{tab:CRediT}
\renewcommand{\arraystretch}{1.0}
\begin{tabular}{|c|>{\raggedright}m{3cm}<{\baselineskip7pt}|m{115mm}<{\baselineskip7pt}|}
\hline
Код & возможная формулировка в статье\centering & значение в онтологии\centering\arraybackslash\\
\hline
1 & идея &\---концептуализация идеи; формулирование или эволюция всеобъемлющих целей и задач исследования\\
2 & методология &\---разработка или проектирование методологии исследований; создание моделей\\
3 & программирование &\---программирование, разработка программного обеспечения; разработка компьютерных программ; реализация компьютерного кода и вспомогательных алгоритмов; тестирование существующих компонентов кода\\
4 & валидация &\---проверка общей воспроизводимости результатов/экспериментов и других результатов исследований.\\
5 & формальный анализ &\---применение статистических, математических, вычислительных или других формальных методов для анализа или синтеза данных исследования.\\
6 & проведение экспериментов &\---проведение процесса исследования и расследования, в частности проведение экспериментов или сбор данных/доказательств.\\
7 & сбор материала &\---предоставление исследовательских материалов, реагентов, материалов, пациентов, лабораторных образцов, животных, инструментов, вычислительных ресурсов или других инструментов анализа.\\
8 & курирование данных &\---действия по управлению для аннотирования (производства метаданных), очистки данных и поддержания исследовательских данных (включая программный код, где это необходимо для интерпретации самих данных) для первоначального использования и последующего повторного использования.\\
9 & написание черновой версии &\---подготовка, создание и/или представление опубликованной работы, особенно написание первоначального черновика (включая основной перевод)\\
10 & доработка и редактирование &\---подготовка, создание и/или представление опубликованной работы участниками исходной исследовательской группы, особенно критический обзор, комментарий или пересмотр, включая этапы до или после публикации.\\
11 & визуализация данных &\---подготовка, создание и/или представление опубликованной работы, в частности визуализация/представление данных.\\
12 & надзор &\---надзор и ответственность руководства за планирование и выполнение исследовательской деятельности, включая наставничество вне основной группы.\\
13 & администрирование проекта &\---ответственность за управление и координацию планирования и выполнения исследовательской деятельности.\\
14 & финансирование &\---Организация финансовой поддержки\\
\hline
\end{tabular}
\vskip -3cm
\end{table}
\end{landscape} \noindent % конец вставки
\newline\noindent
 \Href{https://www.elsevier.com/authors/policies-and-guidelines/credit-author-statement}{Elsevier} и \Href{https://onlinelibrary.wiley.com/doi/epdf/10.1002/leap.1210}{Wiley}.
Команда  \verb|\CRediT{|...\verb|}{|...\verb|}| имеет аргументами
 список кодов формулировок и процент участия автора.
Если вклад авторов различается, то авторы могут использовать для описания индивидуального вклада как это показано \hyperlink{AuthorContfibution}{в конце русской части этого текста}.
Формулировка кода может быть уточнена в квадратных скобках. Например,
\verb|\CRediT{1,2,3[создание стилевого файла],7,9,10,11,13}{95\%}| задаёт последний абзац русскоязычной части этого pdf.

\par\pagegoal=\vsize % восстановление разрушенного макропакетом pdflscape межстрочного интервала по завершению абзаца с ландшафтной вставкой

Если кто-то из авторов спонсируется односторонне заинтересованным в результатах исследования источником), то необходимо сформулировать в аргументе команды \verb|\ConflictOfInterests{|...\verb|}| всю информацию о конфликтах интересов авторов.

\section{Структура и текст статьи}\label{sec:text}
\subsection{Стиль статьи}
Стиль статей в журнале «Программные системы: теория и приложения» нацелен на ясность восприятия.

Безличные предложения (passive voice) считаются дурным доном в~англоязычных издательствах.
Современный академический стиль вместо «Проведён анализ» предпочитает «Мы провели анализ» если авторов несколько либо
анализ приведён выше, что позволяет под «Мы» подразумевать автора с читателем. Иначе «Проведённый анализ» должен стать началом предложения.

Предложения должны быть полными, законченными, не слишком длинными, но не должны начинаться формулой, цифрой или малознакомой аббревиатурой.
Желательно исключать слова и части фраз, которые не способствуют лучшему пониманию результата.

Редакция просит авторов придерживаться единообразного разумного стиля использования буквы ё (либо всюду, либо там, где это существенно для понимания)
и обязательности/ненужности дублирования знаков математических операций при переносе части формулы на новую строку в русской версии.

Русские буквы в формулах стандартно выглядят прямыми в отличие от английских.

\subsection{Списки}
Класс \verb|psta.cls| использует пакет \verb|enumitem| для разнообразных стилей перечислений \Href{http://psta.psiras.ru/read/psta2017_1_3-46.pdf\#page=38}{с таким, например, результатом}.
Стиль перечисления указывается в первой строке, влияет на вид списка и соответственно оформляет ссылки \verb|\ref{|метка\verb|}| на помеченные командой \verb|\label{|метка\verb|}| элементы списка.
Например, оформление

\verb|\begin{enumerate}[1.1]|

\noindent
задаёт нумерацию списков  вида 1, 1.1, 2, 2.1, 2.1.1, 2.1.2, оформление \verb|[abvgd]| задаёт список с нумерацией русскими буквами, \verb|[abc]|\---латинскими буквами, заглавными буквами нумеруют \verb|[ABVGD]| и \verb|[ABC]|.

\subsection{Таблицы и иллюстрации}
Рисунки и таблицы оформляются окружениями \verb|figure| и \verb|table| без модификаторов \verb|[ht]|.
Рисунки и таблицы всегда сами в англоязычной версии располагаются вверху страницы, а в русскоязычной версии строго следуют по тексту, как того требует ГОСТ по отчётам НИР.
Это автоматически срабатывает, если вставка в исходном файле непосредственно следует за её первым упоминанием в тексте, порой разбивая абзац или предложение.
Каждый рисунок сопровождается лаконичным подзаголовком (а таблица \--- заглавием) и меткой \verb|\label|, на которую обязательно должна указывать ccылка \verb|\ref| из описания в тексте.

Пакет \verb|booktabs| улучшает вид таблиц в англоязычных текстах, в русскоязычных обрамление полное.
Ландшафтное расположение больших таблиц и рисунков \Href{http://psta.psiras.ru/read/psta2017_1_121-134.pdf\#page=10}{как здесь} обеспечивает пакет \verb|pdflscape|.

Иллюстрации к статье должны иметь высокое качество, не страдающее при семикратном увеличении, а размер и шрифт надписей на рисунках должен соответствовать оформлению основного текста.
Для сложных иллюстраций с подрисунками полезен пакет \verb|subcaption|, определяющий окружение \verb|subfigure|. В нём синтаксис команд \verb|\caption| и \verb|\label| такой же как в окружении \verb|figure|. Это обеспечивает автоматические ссылки на подрисунки и нумерацию их в русском тексте русскими буквами.
Результат можно посмотреть \Href{http://psta.psiras.ru/read/psta2017_1_135-149.pdf\#page=5}{здесь}.

Иллюстрации используются в формате, обеспечивающем типографское качество:
\begin{itemize}
\item для схем, диаграмм и графиков обязательна  векторная графика (файлы из офисных приложений сохраняются в PDF);
\item снимки с экрана делаются при максимально доступном размере окна на дисплее с высоким разрешением и сохраняются в PNG;
\item фотографии делаются с высоким разрешением в JPG, потери сжатия не должны быть заметны при внимательном просмотре.
\end{itemize}


\subsection{Тире и перенос в словах с дефисом}\label{sec:auto}
Имеющиеся в макропакете \texttt{babel} специфики национальных типографских традиций стилевой файл дополняет макрокомандами, срабатывающими корректно в любом языке.

При использовании стилевого файла действуют макрокоманды пакета \texttt{ncs} для тире в предложениях и составных словах:
\begin{itemize}
\item макрокоманда русского тире \verb|\---| в русском тексте даёт более короткое тире, чем в английском.
\item использование \verb|\=/| вместо \verb|-| в словосочетании
 <<сложно\=/сочинённое>> и других составных словах разрешает заблокированное автоматические переносы частей слова с дефисом на другую строчку, запрещая разрыв по самому тире.
Для разрешения разрыва также и по тире используется \verb|\-/|.
\end{itemize}


\subsection{Оформление списка литературы и гиперссылки}
Возможность перехода по ссылке нередко облегчает читателю восприятие текста.

Стилевой файл предлагает широкое использования в русском тексте и макрокоманду \verb|\Altref{|текст\verb|}{|метка\verb|}|, упрощающую переход читателя по гиперссылке на рисунок, таблицу или раздел поскольку курсор мышки легче установить на слово, чем на цифру.
Пример в конце введения этого шаблона \verb|\Altref{В~разделе}{sec:meta}| делает чувствительным к клику мыши текст \emph{В~разделе}.

Использование пакета \Href{https://www.mathnet.ru/poffice/amsbibpackage.phtml?wshow=amsbibpackage&option_lang=rus}{amsbib} при оформлении списка литературы позволяет единообразно оформить русскоязычный список литературы в~ГОСТ и англоязычный в духе AMS (как ниже).
В англоязычном списке в отличие от ГОСТ инициалы приводятся перед фамилией в списках, сформированных в порядке цитирования.
Авторам достаточно предоставить русскоязычный список с переводами заголовков книг и статей в комментариях по образку исходного файла этого документа, тогда англоязычный в редакции сформируется автоматически.

Рекомендуется ссылки на сайты по возможности не выносить в список литературы, а оформлять на месте или в подстрочных примечаниях гиперссылками \verb|\Href{ссылка}{текст}| с описанием источника. Для ссылок, в порядке исключения включаемых в список литературы, кроме самой ссылки обязательно указывается организация \verb|\publ| и заголовок  \verb|\eprint|, а если на странице указаны, то авторы \verb|\by| и год \verb|\yr|.

\section*{Заключение}


Об ошибках компиляции шаблона и нерешённых проблемах реализации желанного оформления просим незамедлительно сообщить автору по e-mail
\href{mailto:svz@latex.pereslavl.ru}{svz@latex.pereslavl.ru}.

\begin{thebibliography}{9}

\RBibitem{Safonov2007}
\by Сафонов В.\,О.
\paper Молодым программистам: как писать научные работы по ИТ
%\paper For young programmers: how to write scientific papers on IT
%\lang{in Russian}
\jour Компьютерные инструменты в образовании
\publaddr СПб
\publ Изд-во ЦПО «Информатизация образования»
\yr 2007
\issue 6
\pages 11--22
\elibrary 13503158

\RBibitem{Fradkov2003}
\by Фрадков В.\,О.
\paper Как опубликовать хорошую статью и отклонить плохую. Заметки рецензента
%\paper How to publish a good article and reject a bad one. Reviewer's notes
%\lang{in Russian}
\jour Автоматика и телемеханика
\yr 2003
\issue 10
\pages 149--157
\doi https://doi.org/10.1023/1026025826125

\RBibitem{Sviderskaya2011}
\eprint Как написать и опубликовать статью в международном научном журнале
%\eprint How to write and publish an article in an international scientific journal
\eprintinfo Сост. И.\,В.~Свидерская, В.\,В.~Кратасюк
\publaddr Красноярск
\publ Сиб. федерал. ун-т
\yr 2011
\totalpages 52
\URL http://index.petrsu.ru/files/Kak_napisat_i_opublikovat_statyu.pdf

\RBibitem{KirillovaLong}
\eprint Методические рекомендации по подготовке и оформлению научных статей в журналах, индексируемых в международных наукометрических базах данных
\publ Ассоциация научных редакторов и издателей
\publaddr М.
\eprintinfo под общ. ред.  О.\,В.~Кирилловой
\yr 2017
\totalpages 144
\miscnote Прил.
\elibrary 36503266

\RBibitem{KirillovaShort}
\paper Краткие рекомендации для авторов по подготовке и оформлению научных статей в журналах, индексируемых в международных наукометрических базах данных
\by Кириллова~О.\,В., Парфенова~С.\,Л., Гришакина~Е.\,Г., Кулешова~А.\,В., Базанова~Е.\,М., Доронина~Е.\,Г., Зельдина~М.\,М., Безроднова~К.\,А.
\jour Лучевая диагностика и терапия
\yr 2017
\issue 1(8)
\pages 6-12
\elibrary 29004820

\RBibitem{Stolarov}
\by Столяров~А.\,В.
\book Сверстай диплом красиво: LaTeX за три дня
\publ МАКС Пресс
\publaddr М.
\yr 2010
\totalpages  98
\URL http://www.stolyarov.info/books/pdf/latex3days.pdf

\RBibitem{Vorontsov}
\by  Воронцов К.\,В.
\eprint LaTeX2e в примераx
\totalpages  59
\URL http://www.ccas.ru/voron/download/voron05latex.pdf

\RBibitem{Syutkin}
    Сюткин В. \LaTeXe\ документация на русском языке. 2002.~\--- 145 с\URL http://www-sbras.nsc.ru/win/docs/TeX/LaTex2e/docs_koi.html

\Bibitem{AMSshort} Downes~M., Beeton~M. Short Math Guide for LATEX. \foreignlanguage{english}{American Mathematical Society}, 2017

\bibitem{Mathmode} Vo\ss~H.,  Mathematical Typesetting with \LaTeX, 2017
    \URL http://mirror.ctan.org/info/math/voss/mathmode/Mathmode.pdf

\bibitem{CRediT} Liz A., O’Connell A., Kiermer V. How can we ensure visibility and diversity in researchcontributions? How the Contributor Role Taxonomy(CRediT) is helping the shift from authorship tocontributorship. // Learned Publishing 2019 v. 32. Pp. 71–74 \crossref{10.1002/leap.1210}

\end{thebibliography}
\AuthorImageWidth{19mm}
\makefinish

\selectlanguage{english}
\maketitle
\begin{thebibliography}{9}


\bibitem{Safonov2007}
\by V.\,O.~Safonov
\paper For young programmers: how to write scientific papers on IT
\lang{in Russian}
\jour Komp'juternye instrumenty v obrazovanii
\publaddr SPb
\publ Izd-vo CPO «Informatizacija obrazovanija»
\yr 2007
\issue 6
\pages 11--22
\elibrary 13503158

\bibitem{Fradkov2003}
\by V.\,O.~Fradkov
\paper How to publish a good article and reject a bad one. Reviewer's notes
\lang{in Russian}
\jour Avtomatika i telemehanika
\yr 2003
\issue 10
\pages 149--157
\doi https://doi.org/10.1023/1026025826125

\bibitem{Sviderskaya2011}
\eprint How to write and publish an article in an international scientific journal
\eprintinfo Sost. I.\,V.~Sviderskaja, V.\,V.~Kratasjuk
\publaddr Krasnojarsk
\publ Sib. federal. un-t
\yr 2011
\totalpages 52
\URL http://index.petrsu.ru/files/Kak_napisat_i_opublikovat_statyu.pdf

\bibitem{KirillovaLong}
\eprint Guidelines for the preparation and design of scientific articles in journals indexed in international scientometric databases
\publ Associacija nauchnyh redaktorov i izdatelej
\publaddr M.
\eprintinfo pod obshh. red.  O.\,V.~Kirillovoj
\yr 2017
\totalpages 144
\lang{in Russian}
\elibrary 36503266

\bibitem{KirillovaShort}
\paper Brief recommendations for authors on the preparation and design of scientific articles in journals indexed in international scientometric databases
\by Kirillova~O.\,V., Parfenova~S.\,L., Grishakina~E.\,G., Kuleshova~A.\,V., Bazanova~E.\,M., Doronina~E.\,G., Zel'dina~M.\,M., Bezrodnova~K.\,A.
\jour Luchevaja diagnostika i terapija
\yr 2017
\issue 1(8)
\pages 6-12
\lang{in Russian}
\elibrary 29004820

\bibitem{Stolarov}
\by Stoljarov~A.\,V.
\book Type up a diploma beautifully: LaTeX in three days
\publ MAKS Press
\publaddr M.
\yr 2010
\totalpages  98
\URL http://www.stolyarov.info/books/pdf/latex3days.pdf

\bibitem{Vorontsov}
\by K.\,V.~ Voroncov
\eprint LaTeX2e in examples
\lang{in Russian}
\totalpages  59
\URL http://www.ccas.ru/voron/download/voron05latex.pdf

\bibitem{Syutkin}
    Sjutkin V. \LaTeXe\ dokumentacija na russkom jazyke. 2002.~\--- 145 s\URL http://www-sbras.nsc.ru/win/docs/TeX/LaTex2e/docs_koi.html

\bibitem{CRediT} Liz A., O’Connell A., Kiermer V. How can we ensure visibility and diversity in research contributions? How the Contributor Role Taxonomy(CRediT) is helping the shift from authorship to contributorship. // Learned Publishing 2019 v. 32. Pp. 71–74 \crossref{10.1002/leap.1210}

\end{thebibliography}

\end{document}
ToDo: > Хочется, чтобы в статье при обсуждении тех или иных приемов были примеры с результатом, чтобы читатель мог сам посмотреть их исходник. Но когда Вы лишь даете в качестве примера ссылку на некий pdf в сети только с результатом (как в разделах 3.2 и 3.3), у читателя нет возможности сопоставить этот результат с исходником, чтобы понять, как это сделано.
