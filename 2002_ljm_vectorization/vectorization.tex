\documentclass[
11pt,%
tightenlines,%
twoside,%
onecolumn,%
nofloats,%
nobibnotes,%
nofootinbib,%
superscriptaddress,%
noshowpacs,%
centertags]%
{revtex4}
\usepackage{ljm}
\usepackage{listings}
%\usepackage{mathacent}

\lstset{
language=C++,
basewidth=0.5em,
xleftmargin=45pt,
xrightmargin=45pt,
basicstyle=\small\ttfamily,
keywordstyle=\bfseries\underbar,
numbers=left,
numberstyle=\tiny,
stepnumber=1,
numbersep=10pt,
showspaces=false,
showstringspaces=false,
showtabs=false,
frame=trBL,
tabsize=2,
captionpos=t,
breaklines=true,
breakatwhitespace=false,
escapeinside={\%*}{*)}
}

\begin{document}

\titlerunning{Flat Loops Vectorization}
\authorrunning{A.~A.~Rybakov and ...}

\title{Vectorization of Flat Loops of Arbitrary Structure\\Using Instructions AVX-512}

\author{\firstname{A.~A.}~\surname{Rybakov}}
\email[E-mail: ]{rybakov.aax@gmail.com}
\affiliation{Joint Supercomputer Center of the Russian Academy of Sciences -- branch of Scientific Research Institute of System Analysis of the Russian Academy of Sciences, Leninsky prospect 32a, Moscow, 119334, Russia}

\firstcollaboration{(Submitted by S.~S.~Submitter)}

\received{April 01, 2020}

\begin{abstract}
Abstract.
\end{abstract}

\subclass{} % Enter 2010 Mathematics Subject Classification.

\keywords{Keyword1, keyword2.}

\maketitle

\section{Introduction}

Introduction.

\section{AVX-512 instruction set}

AVX-512 instruction set.
    
\begin{table}[!h]
\setcaptionmargin{0mm}
\onelinecaptionsfalse
\captionstyle{flushleft}
\caption{AVX-512 vector single precision float instructions for flat loops vectorization and their semantic.}
\bigskip
\begin{tabular}{|c|c|}
\hline
\textit{instruction name} & \textit{instrucition semantic} \\
\hline
\parbox{10cm}{VMOVAPS, VMOVUPS, VSQRTPS, VGETEXPPS, VGETMANTPS, VRCP14PS, VREDUCEPS, VRNDSCALEPS, VRSQRT14PS, VSCALEFPS} & $\begin{matrix} R = (op \ A) \\ R = \check{P} \ ? \ (op \ A) : R \\ R = \check{P} \ ? \ (op \ A) : 0 \end{matrix}$ \\
\hline
\parbox{10cm}{VADDPS, VANDPS, VANDNPS, VDIVPS, VMAXPS, VMINPS, VMULPS, VORPS, VSUBPS, VRANGEPS} & $\begin{matrix} R = op \ A, B \\ R = \check{P} \ ? \ (op \ A, B) : R \\ R = \check{P} \ ? \ (op \ A, B) : 0 \end{matrix}$ \\
\hline
\parbox{10cm}{VFMADD132PS, VFMADD213PS, VFMADD231PS, VFMSUB132PS, VFMSUB213PS, VFMSUB231PS, VFNMADD132PS, VFNMADD213PS, VFNMADD231PS, VFNMSUB132PS, VFNMSUB213PS, VFNMSUB231PS} & $\begin{matrix} R = (op \ R, A, B) \\ R = \check{P} \ ? \ (op \ R, A, B) : R \\ R = \check{P} \ ? \ (op \ R, A, B) : 0 \end{matrix}$ \\
\hline
\parbox{10cm}{VCMPPS (all variety of compare instructions)} & $\begin{matrix} \check{P} = op \ A, B \\ \check{P} = \check{Q} \ ? \ (op \ A, B) : 0 \end{matrix}$ \\
\hline
\parbox{10cm}{VBLENDMPS} & $\begin{matrix} R = \check{P} \ ? \ A : B \end{matrix}$ \\
\hline
\end{tabular}
\label{tab:avx512instructions}
\end{table}   
    
\section{Flat loops vectorization techniques}

Flat loops vectorization techniques.

\section{Flat loops vectorization practical examples}

Flat loops vectorization practical examples.

\subsection{Riemann solver vectorization}

Riemann solver vectorization.

\subsection{Vectorization of cells classification in the implementation of the immersed boundary method}

Vectorization of cells classification in the implementation of the immersed boundary method.

\section{Conclusion}

Conclusion.

\begin{acknowledgments}
The work has been done at the JSCC RAS as part of the state assignment for the topic ... The supercomputer MVS-10P, located at the JSCC RAS, was used for calculations during the research.
\end{acknowledgments}

\begin{thebibliography}{99}

\bibitem{Rettinger}
\refitem{article}
C. Rettinger, C. Godenschwager, S. Eibl, et al., {\it ``Fully Resolved Simulations of Dune Formation in Riverbeds"}, ISC High Performance , LNCS~{\bf 10266}, 3--21 (2017).

\end{thebibliography}

\end{document}
