{\bf Theorem}. Algorithm of the type (\ref{eqn:eqn_1}, \ref{eqn:eqn_2}) for DO type identification is stable.

{\bf Evidence}. Directly follows from the following facts:

-- The coordinates of the elevation angle, azimuth and range of the detected DO measured by OED are initially considered to be minimal sufficient statistics.
Measurements are carried out on the final sliding period of time.
Collections of dimensions are represented by vectors of finite-dimensional spaces.

-- Samples of measurements of coordinates on the goal-like images generated by the OED are filtered and not transmitted to the DO recognition algorithm \cite{bib_01}.
 
-- Minimum sufficient statistics are independent both in their totality and in pairs on each of the time intervals for detecting DO, subject to the same law of probability distribution.

-- The sets of minimal sufficient statistics make up samples of a sufficiently large volume, they are converted into sufficient statistics as input data for the recognition algorithm of the type of detected DO.
Such statistics for the algorithm, as noted above, are: the fractal dimension of the sample, the energy of the wavelet spectrum of the sample and the maximum eigenvalue of the sample offset correlation matrix.

-- The property of sufficiency of statistics for the identification algorithm is due to the fact that their calculation is performed by algorithmically implemented linear functionals and operators \cite{bib_01}.
 
Thus, the calculation of the fractal dimension is fundamentally based on a power law by minimizing the linear functional, represented by the sum of the squared deviations of the sample data from the true ones.

The calculation of the maximum eigenvalue is performed on the basis of the solution of the characteristic equation for the sample offset correlation matrix as a linear operator. 

The energy of the wavelet spectrum of the sample is a linear transformation in the form of a sum of squares of the decomposition wavelet coefficients for measurement samples according to basic orthogonal wavelets with the corresponding shift and scale parameters, while the wavelet coefficients are determined by scalar products (linear functionals) of the sampling process on basic orthogonal wavelets.

-- DL type recognition is carried out by a team of likelihood ratio criteria and is reduced to performing a comparison operation of the statistics generated by it with optimally established threshold levels of the form of the right-hand side (\ref{eqn:eqn_2}).

-- DO type recognition statistics are sufficient: they are functionals of likelihood ratios of simple DL type hypotheses, directly realized by the collective by the recognition algorithm. Likelihood functions are beta distributions, their parameters are stable.

-- Likelihood functionals are defined in finite-dimensional space, so they are bounded and continuous, that is, the likelihood functionals are linear \cite{bib_17}.

In this regard, as proved \cite{bib_11}, the algorithm that implements linear functionals is stable.

This statement completes the proof of the stability of the DO type identification algorithm -- the algorithm for implementing the collective criteria (\ref{eqn:eqn_1}).
