It is established the invariance of the structure of the wavelet-fractal-correlation DO type identification algorithm to the peculiarities of the current phono-target environment controlled by OED.

The statistical stability of the initial sufficient statistics for the algorithm and the statistical stability of DO type identification algorithm itself are proved for various features of its flight path and sight.

The performance indicators of the wavelet-fractal-correlation type recognition algorithm are estimated for various real-world operating conditions of the OED as the primary sensor of the sets of measurements of the coordinates of the elevation angle, azimuth and range of the detected DO at each finite current time interval and the stable performance of the algorithm in both simple and complex phono-target conditions for the functioning of the OED, including during intensive maneuvering of DO.

Set forth above it represents the development of the theory of methods for assessing the stability of the functioning of algorithms in the conditions of statistical and tactical uncertainty about the current phono-target situation in the zone of control of OED.


\begin{acknowledgements} The work was done at the Joint Supercomputer Center of the Russian Academy of Sciences - Branch of Federal State Institution Scientific Research Institute for System Analysis of the Russian Academy of Sciences within the framework of the state assignment (research topic: 065-2019-0014 (reg. no. AAAA-A19-119011590097-1)) and at the Tver State University within the framework of the state assignment (research topic: 2.1777.2017/4.6).
\end{acknowledgements}