%%
%% ****** ljmsamp.tex 13.06.2018 ******
%%
\documentclass[
11pt,%
tightenlines,%
twoside,%
onecolumn,%
nofloats,%
nobibnotes,%
nofootinbib,%
superscriptaddress,%
noshowpacs,%
centertags]%
{revtex4}
\usepackage{ljm}
\usepackage{listings}
\usepackage{amsmath,amsthm,amscd,amsfonts,amssymb}
\usepackage{textcomp}
\usepackage{graphicx}
\usepackage{esvect}
\usepackage{physics}

\lstset{
language=C++,
basewidth=0.5em,
xleftmargin=45pt,
xrightmargin=45pt,
basicstyle=\small\ttfamily,
keywordstyle=\bfseries\underbar,
numbers=left,
numberstyle=\tiny,
stepnumber=1,
numbersep=10pt,
showspaces=false,
showstringspaces=false,
showtabs=false,
frame=trBL,
tabsize=2,
captionpos=t,
breaklines=true,
breakatwhitespace=false,
escapeinside={\%*}{*)}
}

\begin{document}

\titlerunning{Dynamic objects type identification algorithm robustness} % for running heads
\authorrunning{A.~N.~Katulev, A.~N.~Sotnikov, V.~K.~Kemaikin, I.~V.~Kozhukhin} % for running heads
%\authorrunning{First-Author, Second-Author} % for running heads

\title{Robustness of the algorithm of identification of the type of dynamic object found at the finite sequence of 2D background frames of the optoelectron device}

% Splitting into lines is performed by the command \\
% The title is written in accordance with the rules of capitalization.

\author{\firstname{A.~N.}~\surname{Katulev}}
\email[E-mail: ]{katuleva@mail.ru}
\affiliation{Central Research Institute of Aerospace Defense Forces of Ministry of Defense of the Russian Federation, Afanasiy Nikitin Embankment, building 32, Tver, 170026, Russia}

\author{\firstname{A.~N.}~\surname{Sotnikov}}
\email[E-mail: ]{asotnikov@jscc.ru}
\affiliation{Joint Supercomputer Center of the Russian Academy of Sciences - branch of Scientific Research Institute of System Analysis of the Russian Academy of Sciences, Leninsky prospect 32a, Moscow, 119334, Russia}

\author{\firstname{V.~K.}~\surname{Kemaikin}}
\email[E-mail: ]{vk-kem@mail.ru}
\affiliation{State Technical University, Lenin Avenue, building 25, Tver, 170023, Russia}

\author{\firstname{I.~V.}~\surname{Kozhukhin}}
\email[E-mail: ]{kozhukhin@mail.ru}
\affiliation{State Technical University, Lenin Avenue, building 25, Tver, 170023, Russia}

%\noaffiliation % If the author does not specify a place of work.

\firstcollaboration{(Submitted by TODO : SUBMITTER)} % Add if you know submitter.
%\lastcollaboration{ }

\received{TODO : DATE} % The date of receipt to the editor, i.e. December 06, 2017

\begin{abstract}
Here are proposed robustness characteristics of the algorithm of identification of the type of the dynamic object (DO) and the law of probability distribution of identification sufficient statistics, formed by the algorithm under prior uncertainty.
The law is applied for verification and validation of the algorithm.
Wavelet fractal correlation algorithm (WFCA) implements vectorial criterion of ratio of likelihood functions of simple alternative hypotheses - types of DOs, this criterion being invariant to specific features of DO motion trajectories.
The likelihood functions are reconstructed by simulation according to sufficiently representative complexes of implementations of fractal dimensions, energies, wavelet spectra and maximum eigenvalues of biased correlation matrices as functional of the measured coordinates of spatial attitude of various types of real DOs located by the optoelectron device (OED).
The simulation proved  robustness and high efficiency of the algorithm of identification of the type of DOs.
\end{abstract}

%\subclass{TODO : CODE} % Enter 2010 Mathematics Subject Classification.

\keywords{identification algorithm, dynamical object, optoelectronic device} % Include keywords separeted by comma.

\maketitle

% Text of article starts here.

\section{Relevance}

\begin{figure}[h]
\setcaptionmargin{5mm}
\onelinecaptionstrue
\includegraphics[width=1.0\textwidth]{pics/fig_1_option_1.pdf}
\captionstyle{normal}\caption{The samples of DO coordinate measuring for different angles of standard flight trajectory their observations by OED. Option 1.}\label{fig:fig_1_option_1}
\end{figure}

\begin{figure}[h]
\setcaptionmargin{5mm}
\onelinecaptionstrue
\includegraphics[width=1.0\textwidth]{pics/fig_1_option_2.pdf}
\captionstyle{normal}\caption{The samples of DO coordinate measuring for different angles of standard flight trajectory their observations by OED. Option 2.}\label{fig:fig_1_option_2}
\end{figure}

\begin{figure}[h]
\setcaptionmargin{5mm}
\onelinecaptionstrue
\includegraphics[width=1.0\textwidth]{pics/fig_1_option_3.pdf}
\captionstyle{normal}\caption{The samples of DO coordinate measuring for different angles of standard flight trajectory their observations by OED. Option 3.}\label{fig:fig_1_option_3}
\end{figure}


\section{The objective of article}

Substantiation of the stability property of the initial data as sufficient statistics for WFCA identification of DO type, the stability of the identification statistics generated by the algorithm and the stability of the algorithm itself.
Identifying the probability distribution law for recognition statistics and its stability properties.
Estimation of efficiency indicators for  the DO type identification with a priori uncertainty regarding the phono-target situation in the zone of control of the OED.

\section{Wavelet-fractal-correlation algorithm: the structure and properties}

The structure and computational operations of VFKA are completely determined by expressions of DO type identification criteria based on sufficient statistics: fractal dimensions -- $d(\cdot)$, wavelet spectral energies -- $u(\cdot)$ and maximum eigenvalues -- $\lambda(\cdot)$ for shifted correlation matrices of samples of elevation coordinates measurements for elevation angle -- $\theta$, azimuth -- $\phi$ and elevation angle -- $D$ of DO locating OED (the dot $\cdot$ is a measured coordinate, three types of DO have nine statistics).
 
The sufficiency of each of these statistics is physically determined by the fact that they are calculated directly from samples of DO coordinate measurements as minimal sufficient statistics, and formally it is established by the well-known Rao-Blequella-Kolmogorov theorem: optimal estimates (subject to their existence) are functions of sufficient statistics \cite{bib_07}.
In this case, the validity of the application of the theorem is obvious.
Additionally, we note that in \cite{bib_01}, the authors confirmed the sufficiency property based on the analysis of the likelihood functions of obtaining measurement samples under the condition that they are located in the zone of control of the OED for DO corresponding type which we should recognize (as an option it is considered transport aircraft, fighter and helicopter \cite{bib_08}).
Identification criteria are written as \cite{bib_01} 

\begin{equation}\label{eqn:eqn_1}
\begin{gathered}
\frac{f_d(d(\theta) | p_j, q_j, d(\theta_j), s_j)}{f_d(d(\theta) | p_i, q_i, d(\theta_i), s_i)} \ge 1,
\frac{f_d(d(\phi) | p_j, q_j, d(\phi_j), s_j)}{f_d(d(\phi) | p_i, q_i, d(\phi_i), s_i)} \ge 1,
\frac{f_d(d(D) | p_j, q_j, d(D_j), s_j)}{f_d(d(D) | p_i, q_i, d(D_i), s_i)} \ge 1, \\
\frac{f_u(u(w(\theta)) | p_j, q_j, u(w_j), s_j)}{f_u(u(w(\theta)) | p_i, q_i, u(w_i), s_i)} \ge 1,
\frac{f_u(u(w(\phi)) | p_j, q_j, u(w_j), s_j)}{f_u(u(w(\phi)) | p_i, q_i, u(w_i), s_i)} \ge 1,
\frac{f_u(u(w(D)) | p_j, q_j, u(w_j), s_j)}{f_u(u(w(D)) | p_i, q_i, u(w_i), s_i)} \ge 1, \\
\frac{f_{\lambda}(\lambda(\theta) | p_j, q_j, \lambda(\theta_j), s_j)}{f_{\lambda}(\lambda(\theta) | p_i, q_i, \lambda(\theta_i), s_i)} \ge 1,
\frac{f_{\lambda}(\lambda(\phi) | p_j, q_j, \lambda(\phi_j), s_j)}{f_{\lambda}(\lambda(\phi) | p_i, q_i, \lambda(\phi_i), s_i)} \ge 1,
\frac{f_{\lambda}(\lambda(D) | p_j, q_j, \lambda(D_j), s_j)}{f_{\lambda}(\lambda(D) | p_i, q_i, \lambda(D_i), s_i)} \ge 1,
\end{gathered}
\end{equation}
 
where $f_d(\cdot | \cdot)$, $f_u(\cdot | \cdot)$, $f_{\lambda}(\cdot | \cdot)$ are beta-likelihood functions of samples, provided that they are in the control zone of OED for DO type $s_j$, $j,i = 1,2, \dots, M$, $j \ne i$, $p_j > 0$, $q_i > 0$ -- beta-functions parameters.
 
Criteria (\ref{eqn:eqn_1}) is a vector of independent subtests (independence is determined by the independence of the statistics $d(\theta)$, $d(\phi)$, $d(D)$, $u(w(\theta))$, $u(w(\phi))$, $u(w(D))$, $\lambda(\theta)$, $\lambda(\phi)$, $\lambda(D)$).
 
Directly from (1\ref{eqn:eqn_1} it follows that the structures of the criteria are similar, trans-form after logarithmation to the same type of computational expression and completely determine the sequence of computational operations of the algorithm of DO type identification.

Write this expression, for example, for an algorithm that implements the first criterion from (\ref{eqn:eqn_1}):

\begin{equation}\label{eqn:eqn_2}
\begin{gathered}
(p_j - 1) \ln(u(w(\theta)) - \mu_{j0}) + (q_j - 1) \ln(\mu_{j1} - u(w(\theta))) \\
- (p_i - 1) \ln(u(w(\theta)) - \mu_{i0}) - (q_i - 1) \ln(\mu_{i1} - u(w(\theta))) \ge \\
\ge -\ln{\frac{1}{\mu_{j1} - \mu_{j0}}} - \ln{\frac{\Gamma(p_j + q_j)}{\Gamma(p_j)\Gamma(q_j)}} + (p_j - 1) \ln(\mu_{j1} - \mu_{j0}) + (q_j - 1) \ln(\mu_{j1} - \mu_{j0}) \\
+ \ln{\frac{1}{\mu_{i1} - \mu_{i0}}} + \ln{\frac{\Gamma(p_i + q_i)}{\Gamma(p_i)\Gamma(q_i)}} - (p_i - 1) \ln(\mu_{i1} - \mu_{i0}) + (q_i - 1) \ln(\mu_{i1} - \mu_{i0})
\end{gathered}
\end{equation}

where $\mu_{j0} \le u(w(\theta)) \le \mu_{j1}$, $j,i = 1,2, \dots, M$, $j \ne i$, $\mu_{j0}$, $\mu_{j1}$ are left and right limits of range for values of sample sufficient statistics for identification algorithm.
They are given in \cite{bib_01}.

In this inequality expression, the right-hand side is the threshold level of decision making based on subset DO type identification criterion, and the left side is the recognition statistics as a function of a random variable -- sufficient statistics -- on the wavelet-spectrum energy of a measurements sample for the elevation angle of detecting DO in the current time interval.

From (\ref{eqn:eqn_2}) it can be seen that the recognition statistics is represented by a linear combination of logarithmic functions of sufficient statistics $u(w(\theta))$.
 
Such functions are monotonous, they do not distort the sufficiency of the transformed statistics $u(w(\theta))$ \cite{bib_09,bib_10}.
This means that each component of a linear combination is sufficient statistics as a function of sufficient statistics $u(w(\theta))$ and that a linear combination as a whole as statistics of DO type identification for a particular criterion is sufficient statistics, naturally, random, since the sample is finite.
At the same time, we also note that the sufficiency of statistics is a consequence of the sufficiency properties of each initially considered likelihood ratio \cite{bib_09} and that with a sufficiently large sample, the likelihood ratio statistics are consistent and robust \cite{bib_10}.

The established properties of the statistics of the left-hand side of (\ref{eqn:eqn_2}) are sufficient conditions for assessing the stability of a particular DO type identification algorithm.

The necessary conditions are the stability of each initial sufficient statistics for the algorithm: fractal dimension -- $d(\cdot)$ , energy of the wavelet spectrum -- $u(\cdot)$, maximum eigenvalue -- $\lambda(\cdot)$ as a function of a sample of OED measurements of position for identifying DO, otherwise, as functions of minimal sufficient statistics obtained from the DO detection time interval. 

The stability of private algorithms and the initial sufficient statistics for them together determine the stability property of DO type identification algorithm that implements a vector of criteria (\ref{eqn:eqn_1}).


\section{The stability of the initial sufficient statistics for the DO type identification algorithm}

stability 1 TODO

\section{The stability of the recognition algorithm type}

stability 2 TODO

\section{The probability distribution law of statistics recognition of DO type. Stability properties.}

\begin{figure}[h]
\setcaptionmargin{5mm}
\onelinecaptionstrue
\includegraphics[width=1.0\textwidth]{pics/fig_2.pdf}
\captionstyle{normal}\caption{The laws of the distribution of statistics functions of fractal dimension.}\label{fig:fig_2}
\end{figure}

\begin{figure}[h]
\setcaptionmargin{5mm}
\onelinecaptionstrue
\includegraphics[width=1.0\textwidth]{pics/fig_3.pdf}
\captionstyle{normal}\caption{The laws of the distribution of statistics functions of maximum eigenvalue.}\label{fig:fig_3}
\end{figure}

\begin{figure}[h]
\setcaptionmargin{5mm}
\onelinecaptionstrue
\includegraphics[width=1.0\textwidth]{pics/fig_4.pdf}
\captionstyle{normal}\caption{The laws of the distribution of the statistics functions of the energy of the wavelet spectrum.}\label{fig:fig_4}
\end{figure}


\section{Performance indicators of DO type identification algorithm}

The randomness of the sufficient statistics obtained in (\ref{eqn:eqn_2}) can lead to correct and incorrect recognition of DO type for each particular algorithm, and as a whole by the team of DO type identification algorithms.

The quantitative measure of the identification result is the probability of correct and incorrect recognition of DO type ofthe performance indicators for each particular algorithm and the algorithm as a whole.

Due to the impossibility of recovering analytical expressions of the probability distribution laws of sufficient statistics for DO type identification, private algorithms for evaluating, their performance indicators are obtained here by modeling under various conditions of functioning OED (Fig.~\ref{fig:fig_1_option_1}, Fig.~\ref{fig:fig_1_option_2}, Fig.~\ref{fig:fig_1_option_3} illustrate such conditions).

Private algorithms operate independently of one another.
The output of each of them is represented in the form $\delta_j = 1$ or $\delta_j = 0$, respectively, when recognizing DO type $j$ or making a decision on recognizing DO of another type.
At the same time, the performance indicators in general of the DO type recognition algorithm $j$ will be determined by the binomial distribution, which describes that, of the full team of private algorithms, "voted" at least to private algorithms "for" DO type.

Estimates of the probability of correct identification of DO type by a particular algorithm are obtained by modeling for each sufficiently separately considered statistics; their values $P(s_j | s_j) \cong 0.67$, $j = 1, 2, \dots, 9$, ($s_j$ -- DO type notation $j$, $P(s_j | s_j)$ -- is the probability that the private algorithm makes decisions about the detection of the DO of type $s_j$, when the DO of $s_j$ type really does exist in the control zone of the OED).

The performance indicators of the team of recognition algorithms of the type of detectable DO by OED: probability of correct recognition of DO type 

\begin{equation}
P(s_j) = \sum_{l = 6}^{l = 9}{C_{9}^{l}P^l(s_j | s_j)(1 - P(s_j | s_j))^{9 - l}} \cong 0.6
\end{equation}

where $P(s_j | s_j) \cong 0.67$, $j = 1, 2, \dots, 9$, and the threshold level of making the right decision by a team of algorithms, as established in \cite{bib_01}, is equal to $6$; 

probability of type confusion for DO type $P(s_j) = 1 - P(s_j | s_j) \cong 0.4$

The probability value $P(s_j)$, as noted above, can be increased either by using a priori information about the current phono-target situation in the zone of control of the OED, but only by relatively simple stationary conditions, or by accumulating the results of "voting" by the team of private algorithms at regular time intervals of DO type indefication, implementation of the recognition criterion "at least once $k$ from $l$" for $n$ intervals \cite{bib_20,bib_21} in any in real phono-target conditions.

Here, the second approach and the "$k = 3$ from $l = 5$" criterion are adopted, as the closest to the optimal "$3$ out of $4$" \cite{bib_22}; it is implemented by the authors in \cite{bib_01}.

The decision on DO type is estimated by cumulative probability by the recurrent expression \cite{bib_22} $P(3, 5; n) = P(3, 5; n - 1) + f(3, 5; n)$, where $P(3, 5; n)$ is the probability of the "$3$ out of $5$" criterion at least once at consecutive intervals, $P(3, 5; n - 1)$ -- the same probability over $(n - 1)$ the intervals, $f(3, 5; n)$ the probability of the $3$ out of $5$ criterion execution for the first time on n time interval.

The calculation $P(\cdot)$ and $f(3, 5; n)$ is performed on guaranteed values of the probabilities $P(s_j) \triangleq p_n = P_{pr.rasp} = 0.6$ and $P(s_j) \triangleq q_n = P_{per} = 0.4$ at successive intervals, starting with the first, $n = 1$ , then for $n = 2, 3, 4, \dots$.
  
The results are as follows: with $n = 1$ probability $P(3, 5; n = 1) = 0$, with $n = 2$ $P(3, 5; n = 2) = P_{pr.rasp} \cdot P_{pr.pasr} = 0$, with $n = 3$ $P(3, 5; n = 3) = 0.6^3$, with $n = 4$ $P(3, 5; n = 4) = 0.64$, with $n = 5$ $P(3, 5; n = 5) = 0.84$, with $n > 5$ $P(3, 5; n > 5) > 0.9$ and $P_{per} < 0.1$.
 
Estimates of the probability of correct recognition of DO type are lower guaranteed, the probability of entanglement is upper guaranteed. Note that in \cite{bib_22} the value of the probability of correct recognition of DO type was obtained in the range of $0.708$ -- $0.976$ with the introduction of a priori formed standards of the contours of the detected DO with a simple background.


\section{Conclusion}

It is established the invariance of the structure of the wavelet-fractal-correlation DO type identification algorithm to the peculiarities of the current phono-target environment controlled by OED.

The statistical stability of the initial sufficient statistics for the algorithm and the statistical stability of DO type identification algorithm itself are proved for various features of its flight path and sight.

The performance indicators of the wavelet-fractal-correlation type recognition algorithm are estimated for various real-world operating conditions of the OED as the primary sensor of the sets of measurements of the coordinates of the elevation angle, azimuth and range of the detected DO at each finite current time interval and the stable performance of the algorithm in both simple and complex phono-target conditions for the functioning of the OED, including during intensive maneuvering of DO.

Set forth above it represents the development of the theory of methods for assessing the stability of the functioning of algorithms in the conditions of statistical and tactical uncertainty about the current phono-target situation in the zone of control of OED.


\begin{acknowledgements} The work was done at the Joint Supercomputer Center of the Russian Academy of Sciences - Branch of Federal State Institution Scientific Research Institute for System Analysis of the Russian Academy of Sciences within the framework of the state assignment (research topic: 065-2019-0014 (reg. no. AAAA-A19-119011590097-1)) and at the Tver State University within the framework of the state assignment (research topic: 2.1777.2017/4.6).
\end{acknowledgements}

\begin{acknowledgments}
The work was done
\end{acknowledgments}

\begin{thebibliography}{99}

\bibitem{bib_tag}
\refitem{article}
Author, {\it ``Name"}, Journal Vol.~1, No.~1, 1 -- 100 (2019).

\end{thebibliography}

\end{document}
