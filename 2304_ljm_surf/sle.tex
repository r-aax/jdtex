Основной проблемой, с которой можно потенциально столкнуться при эволюции расчетной сетки, является возникновение самопересечений.
Самопересечение является критическим дефектом сетки, при котором невозможно производить дальнейшие вычисления по моделированию ледообразования, поэтому самопересечения необходимо удалять.
Во-первых, необходимо определить, какое отношение двух ячеек сетки можно трактовать как самопересечение.
Конечно, простое наличие общих точек у двух ячеек не может служить критерием самопересечения, так как у каждой ячейки есть смежные ячейки (то есть две ячейки могут иметь как общую вершину, так и общее ребро).
Так как смежными могут быть только ячейки, имеющие общие инцидентные объекты (общую инцидентную вершину или общее инцидентное ребро), а все отношения инцидентности прописаны в расчетной сетке, то не представляет труда отделить пересечение ячеек по общему инцидентному объекту от самопересечения.

\subsection{Поиск пар пересекающихся треугольников}

В любом случае для идентификации всех фактов самопересечения сетки требуется проанализировать все пары ячеек и проверить пересечение ячеек в паре (то есть проверить пересечение каждой ячейки с каждой).
Так как прямой перебор всех пар ячеек является имеет квадратичную сложность по количеству ячеек в сетке, то его использование на крупных сетках невозможно.
В этом случае целесообразно выполнять поиск пар пересекающихся ячеек с помощью представления множества ячеек в виде специальной древовидной структуры, связанной с ограничением геометрических объектов прямоугольными параллелепипедами в пространстве.
Для начала определим понятие контейнера для произвольного множества точек $M$ в пространстве (будем обозначать его через $[M]$): $[M]$ -- это прямоугольный параллелепипед, являющийся декартовым произведением трех сегментов

\begin{equation}
[M] = \left[\min_{P \in M}{P_x}, \max_{P \in M}{P_x}\right]
      \times \left[\min_{P \in M}{P_y}, \max_{P \in M}{P_y}\right]
      \times \left[\min_{P \in M}{P_z}, \max_{P \in M}{P_z}\right]
\end{equation}

Контейнер можно рассматривать для произвольного множества точек.
Мы будем его использовать для ячейки расчетной сетки, а также для множества ячеек.
Так как любой треугольник является выпуклой фигурой, то $[ABC] = [\{A, B, C\}]$, то есть для построения контейнера для треугольника достаточно рассмотреть только его вершины.
При поиске пар пересекающихся ячеек будем использовать следующий факт: если два треугольника пересекаются, то пересекаются и их контейнеры: $ABC \cap A'B'C' \ne \emptyset \Rightarrow [ABC] \cap [A'B'C'] \ne \emptyset$.
А значит если не пересекаются контейнеры двух треугольников, то не пересекаются и сами треугольники: $[ABC] \cap [A'B'C'] = \emptyset \Rightarrow ABC \cap A'B'C' = \emptyset$.
В отличие от анализа пары треугольников на пересечение, проверка на пересечение двух прямоугольных параллелепипедов со сторонами, параллельными координатным осям, не представляет проблем.
То есть па первом этапе будем искать такие пары ячеек, чьи контейнеры пересекаются (будем называть их потенциально пересекающимися ячейками).

\begin{figure}[h]
\includegraphics[width=0.8\textwidth]{pics/pic_box_size.pdf}
\captionstyle{center}\caption{Схема построения дерева контейнеров.}\label{fig:pic_box}
\end{figure}

Для поиска потенциально пересекающихся ячеек построим дерево контейнеров всех ячеек сетки.
Корнем данного дерева будет являться контейнер всех ячеек расчетной сетки.
Рассмотрим процедуру разделения контейнера на два более мелких на примере двумерного случая, проиллюстрированного на Fig.~\ref{fig:pic_box}.
Путь мы хотим разделить контейнер некоторого множества треугольников ($M$) по горизонтальной прямой на два множества: верхнее ($U$) и нижнее ($D$).
Тогда в верхнее множество попадут все треугольники, лежащие выше прямой, либо пересекающие ее.
Аналогично в нижнее множество попадут все треугольники, лежащие ниже прямой, либо пересекающие ее.
После выделения верхнего и нижнего множества треугольников, для каждого из них строятся свои контейнеры, которые становятся дочерними контейнерами исходного.
Далее получившиеся контейнеры можно делить дальше, используя произвольное направление разбиения ($X$, $Y$, $Z$).
В частности на Fig.~\ref{fig:pic_box} продемонстрировано разбиение исходного контейнера по схеме $[M] \rightarrow \{[U], [D]\} \rightarrow \{\{[UL], [UR]\}, \{[DL], [DR]\}\}$.
Следует заметить, что за одну операцию контейнер можно разбивать на произвольное количество дочерних контейнеров аналогичным образом.

Построенное дерево контейнеров позволяет существенно сократить количество проверяемых потенциальных пересечений треугольников \cite{Jung}, так как если $[M] \cap [M'] = \emptyset$, $[T]$ -- дочерний контейнер для $[M]$, а $[T']$ -- дочерний контейнер для $[M']$, то $[T] \cap [T'] = \emptyset$.

После того, как найдены все пары потенциально пересекающихся треугольников, необходимо проверить, пересекаются ли они на самом деле.
Так как треугольник является выпуклой фигурой, то пересечение двух треугольников также является выпуклой фигурой (это может быть любая плоская фигура с количеством вершин от 1 до 6).
Вершинами пересечения двух треугольников являются точки пересечения сторон одного треугольника с другим треугольником и наоборот.
Таким образом задача поиска пересечения двух треугольников сводится к поиску точек пересечения треугольника и отрезка.
Данная задача может быть решена, с помощью представления треугольника $ABC$ в виде геометрического места точек $\vec{P} = \vec{A} + \beta \vec{AB} + \gamma \vec{AC}$, $\beta \ge 0$, $\gamma \ge 0$, $\beta + \gamma \le 1$, представления отрезка $QR$ в виде геометрического места точек $\vec{P} = \vec{Q} + \phi \vec{QR}$, $0 \le \phi \le 1$ и поиска решения системы уравнений $\vec{A} + \beta \vec{AB} + \gamma \vec{AC} = \vec{Q} + \phi \vec{QR}$ относительно неизвестных $\beta$, $\gamma$, $\phi$ с учетом ограничений \cite{Freylekhman}.

\subsection{Фиксация пересекающихся треугольников}

\begin{figure}
  \centering
  \begin{minipage}[h]{0.32\textwidth}
    \includegraphics[width=\textwidth]{pics/pic_zip_01.png}
    \caption{zip 01.}\label{fig:pic_zip_01}
  \end{minipage}
  \hfill
  \begin{minipage}[h]{0.32\textwidth}
    \includegraphics[width=\textwidth]{pics/pic_zip_09.png}
    \caption{zip 09.}\label{fig:pic_zip_09}
  \end{minipage}
  \hfill
  \begin{minipage}[h]{0.32\textwidth}
    \includegraphics[width=\textwidth]{pics/pic_zip_15.png}
    \caption{zip 15.}\label{fig:pic_zip_15}
  \end{minipage}
\end{figure}

\begin{figure}
  \centering
  \begin{minipage}[h]{0.49\textwidth}
    \includegraphics[width=\textwidth]{pics/pic_self_intersection_on.png}
    \caption{Поверхность до удаления самопересечения.}\label{fig:pic_self_intersection_on}
  \end{minipage}
  \hfill
  \begin{minipage}[h]{0.49\textwidth}
    \includegraphics[width=\textwidth]{pics/pic_self_intersection_off.png}
    \caption{Поверхность после удаления самопересечения.}\label{fig:pic_self_intersection_off}
  \end{minipage}
\end{figure}

\begin{figure}
  \centering
  \begin{minipage}[h]{0.49\textwidth}
    \includegraphics[width=\textwidth]{pics/pic_self_intersection_on_2.png}
    \caption{Поверхность до удаления самопересечения.}\label{fig:pic_self_intersection_on}
  \end{minipage}
  \hfill
  \begin{minipage}[h]{0.49\textwidth}
    \includegraphics[width=\textwidth]{pics/pic_self_intersection_off_2.png}
    \caption{Поверхность после удаления самопересечения.}\label{fig:pic_self_intersection_off}
  \end{minipage}
\end{figure}