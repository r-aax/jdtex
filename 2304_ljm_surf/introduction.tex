Численное моделирование процесса обледенения поверхности тела является сложной мультифизичной задачей, включающей в себя моделирование процессов газовой динамики, теплообмена, течения жидкости, динамики капель в воздушном потоке и других.
Исследование процессов ледообразования имеет важное практическое значение.
В частности характер и интенсивность образования льда на поверхности летательного аппарата критическим образом влияет на его летные характеристики, что напрямую связано с безопасностью полетов \cite{Raj}.

На сегодняшний день среди зарубежного программного обеспечения для моделирования процесса ледообразования лидером является программный комплекс ANSYS (включая модули FENSAP-ICE, DROP3D, ICE3D) \cite{Martini}.
В России также активно ведется разработка математических алгоритмов и программного обеспечения по этому направлению.
Можно отметить исследования по разработке модуля iceFoam в составе открытого пакета OpenFOAM~\cite{Strijhak}.
Среди коммерческих продуктов в последние годы активно развивается пакет IceVision в составе программного комплекса FlowVision~\cite{Sorokin}, а также решение в составе пакета инженерного анализа ЛОГОС \cite{Galanov}.

Моделирование процесса ледяного покрова осуществляется, как правило, на поверхностной расчетной сетке и состоит из двух основных частей.
Первой частью является вычисление интенсивности нарастания льда в отдельных элементах сетки (это может быть вычисление массы скопившегося льда в каждой ячейке расчетной сетки за единицу времени, либо скорость образования ледяного покрова в узлах сетки, либо другие аналогичные характеристики).
Для выполнения вычисления интенсивности нарастания льда в элементах расчетной сетки существует множество моделей ледообразования \cite{Bartkus,Zhang,Pena}, учитывающих разные состояния льда, динамику водяной пленки, тепловые потоки и другие факторы.
Модели ледообразования не рассматриваются в рамках данной работы.
Второй важной составляющей моделирования ледяного нароста является определение изменения поверхности тела после нарастания на ней слоя льда.
В данной статье рассматриваются наиболее известные подходы к моделированию эволюции поверхности обледеневающего тела, а также предлагается новый алгоритм эволюции поверхности, основанный на принципе общей огибающей семейства сфер, центры которых лежат на исходной поверхности, и алгоритм устранения самопересечений сетки, которые могут возникать из-за изменения положения узлов в процессе эволюции.
