Центральная задача перестроения поверхности из-за нарастания ледяного покрова выглядит следующим образом.
Пусть известно, что в результате численного решения задачи ледообразования конечно-объемным методом \cite{Beaugendre} в каждой ячейке сетки была вычислена масса накопленного льда ($m$).
Будем считать плотность льда постоянной, то есть в каждой ячейке также известен объем накопленного льда ($V$).
Для каждого узла сетки $N$ требуется найти его новое положение в пространстве $N'$, чтобы для каждой ячейки с узлами $ABC$ объем пространства, ограниченный фигурой $ABCA'B'C'$ соответствовал объему льда, накопленному в данной ячейке.

Следует отметить, что поставленная задача может не иметь точного решения, и в этом случае следует стремиться к минимизации ошибки по объему (когда фактически образовавшийся объем льда не слишком сильно отличается от целевого объема, то есть разница $V_{ABCA'B'C'} - V$ мала).

Задачу определения новых положений узлов расчетной сетки можно разделить на две задачи: определение направлений смещения узлов и определение величин смещения.
Далее рассмотрим отдельные методы перестроения поверхностей более подробно.
