В статье рассмотрен геометрический аспект задачи моделирования процесса ледообразования, а именно задача эволюции неструктурированной поверхностной расчетной сетки.
В процессе эволюции изменение положения узлов сетки должно соответствовать объему накопленного льда в ячейках сетки, а сама сетка должна оставаться корректной, целостной и не содержать самопересечений.
В противном случае продолжение моделирования процесса ледообразования может оказаться невозможным.

Эволюция расчетной сетки состоит из двух основных этапов.
На первом этапе происходит вычисление новых положений узлов сетки в соответствии с объемом накопленного льда.
Был рассмотрен ряд наиболее распространенных алгоритмов вычисления новых положений узлов, а также предложен новый алгоритм.
Среди рассмотренных алгоритмов вычисления новых положений узлов присутствовали классические алгоритмы явного вычисления координат узлов, базирующиеся на аппроксимации объема льда в виде простых геометрических фигур (метод призм и метод пирамид).
Был рассмотрен многослойный подход, моделирующий итерационное наращивания льда отдельными слоями, позволяющий существенно повысить точность классических методов.
Был рассмотрен алгоритм перестроения поверхности, использующий перемещение вершин с сохранением объема, основными особенностями которого являются пошаговое перестроение, вычисление максимальной доли наращиваемого льда для предотвращения локального самопересечения и применение сглаживаний различного вида для эффективной обработки впадин на поверхности тела.
Был предложен новый алгоритм перестроения поверхности, основанный на формировании новой поверхности в виде общей огибающей семейства сфер, центры которых расположены на исходной поверхности, а радиусы соответствует интенсивности нарастания льда.
Предложенный алгоритм является линейным по количеству ячеек сетки, устойчивым, а также имеет тенденцию к затягиванию небольших трещин и впадин на поверхности и к сглаживанию острых пиков и изломов.

Так как вне зависимости от применяемого алгоритма перестроения поверхности невозможно дать гарантию отсутствия самопересечения результирующей сетки, то в качестве второго этапа эволюции сетки рассматривалось удаление этих самопересечений.
Было рассмотрено два подхода к устранению самопересечений.
В основе обоих подходов лежит поиск пересечений ячеек сетки с другими ячейками, отличными от смежных (ячеек пересечения).
В основе первого подхода устранения самопересечений является удаление ячеек пересечения с последующим восстановлением сетки.
В качестве второго подхода рассматривался метод, основанный на дроблении всех ячеек сетки по точкам пересечения с другими ячейками, обходе расчетной сетки начиная со статической области и удаления после данного обхода всех непомеченных ячеек.

Все рассмотренные методы перестроения поверхностей и устранения самопересечений были реализованы и протестированы на предмет применимости в задачах моделирования ледообразования.
По итогам проведенного тестирования в качестве практического использования был выбран механизм перестроения поверхности, основанный на общей огибающей семейства сфер (так как данный алгоритм является быстрым, надежным и помогает снижать дефекты сетки).
В качестве алгоритма устранения самопересечений сетки предпочтение было отдано алгоритму, основанному на дроблении ячеек по точках пересечения.
Данный алгоритм, хоть и является достаточно медленным (за счет необходимости анализа на пересечение всех потенциально конфликтующих пар ячеек), однако применим к расчетным сеткам произвольной геометрии и сложности.
