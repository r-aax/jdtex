\documentclass[
11pt,%
tightenlines,%
twoside,%
onecolumn,%
nofloats,%
nobibnotes,%
nofootinbib,%
superscriptaddress,%
noshowpacs,%
centertags]%
{revtex4}
\usepackage{ljm}
\usepackage{listings}
\usepackage[utf8]{inputenc}
\usepackage[russian]{babel}

\lstset{
language=C++,
basewidth=0.5em,
xleftmargin=45pt,
xrightmargin=45pt,
basicstyle=\small\ttfamily,
keywordstyle=\bfseries\underbar,
numbers=left,
numberstyle=\tiny,
stepnumber=1,
numbersep=10pt,
showspaces=false,
showstringspaces=false,
showtabs=false,
frame=trBL,
tabsize=2,
captionpos=t,
breaklines=true,
breakatwhitespace=false,
escapeinside={\%*}{*)}
}

\begin{document}

\titlerunning{TODO titlerunning}
\authorrunning{Meshcheryakov et al.}

\title{TOTO title}

\author{\firstname{A.~O.}~\surname{Meshcheryakov}}
\email[E-mail: ]{alex2501@jscc.ru}
\affiliation{Joint Supercomputer Center of the Russian Academy of Sciences -- branch of Scientific Research Institute of System Analysis of the Russian Academy of Sciences, Leninsky prospect 32a, Moscow, 119334, Russia}

\author{\firstname{A.~A.}~\surname{Rybakov}}
\email[E-mail: ]{rybakov@jscc.ru}
\affiliation{Joint Supercomputer Center of the Russian Academy of Sciences -- branch of Scientific Research Institute of System Analysis of the Russian Academy of Sciences, Leninsky prospect 32a, Moscow, 119334, Russia}

\firstcollaboration{(Submitted by TODO)} % Add if you know submitter.
%\lastcollaboration{ }

\received{received TODO}

\begin{abstract}
TODO abstract
\end{abstract}

\subclass{TODO subclass} % Enter 2010 Mathematics Subject Classification.

\keywords{TODO keywords}

\maketitle

%---------------------------------------------------------------------------------------------------

\section{Introduction}
Численное моделирование процесса обледенения поверхности тела является сложной мультифизичной задачей, включающей в себя моделирование процессов газовой динамики, теплообмена, течения жидкости, динамики капель в воздушном потоке и других.
Исследование процессов ледообразования имеет важное практическое значение.
В частности характер и интенсивность образования льда на поверхности летательного аппарата критическим образом влияет на его летные характеристики, что напрямую связано с безопасностью полетов TODO.

На сегодняшний день среди зарубежного программного обеспечения для моделирования процесса ледообразования лидером является программный комплекс ANSYS (включая модули FENSAP-ICE, DROP3D, ICE3D) TODO.
В России также активно ведется разработка математических алгоритмов и программного обеспечения по этому направлению.
Можно отметить исследования по разработке модуля iceFoam в составе открытого пакета OpenFOAM TODO.
Среди коммерческих продуктов в последние годы активно развивается пакет IceVision в составе программного комплекса FlowVision TODO, а также решение в составе пакета инженерного анализа ЛОГОС TODO.

Моделирование процесса ледяного покрова осуществляется, как правило, на поверхностной расчетной сетке и состоит из двух основных частей.
Первой частью является вычисление интенсивности нарастания льда в отдельных элементах сетки (это может быть вычисление массы скопившегося льда в каждой ячейке расчетной сетки за единицу времени, либо скорость образования ледяного покрова в узлах сетки, либо другие аналогичные характеристики).
Для выполнения вычисления интенсивности нарастания льда в элементах расчетной сетки существует множество моделей ледообразования TODO, учитывающих разные состояния льда, динамику водяной пленки, тепловые потоки и другие факторы.
Модели ледообразования не рассматриваются в рамках данной работы.
Второй важной составляющей моделирования ледяного нароста является определение изменения поверхности тела после нарастания на ней слоя льда.
В данной статье рассматриваются наиболее известные подходы к моделированию эволюции поверхности обледеневающего тела, а также предлагается новый алгоритм эволюции поверхности, основанный на принципе общей огибающей семейства сфер, центры которых лежат на поверхности, и алгоритм устранения самопересечений сетки, которые могут возникать из-за изменения положения узлов в процессе эволюции.


%---------------------------------------------------------------------------------------------------

\section{Mesh architecture}
Решение задачи перестроения поверхности будем рассматривать на неструктурированной поверхностной расчетной сетке.
Элементами расчетной сетки являются узлы ($N$), ребра ($e$)  и ячейки ($f$).
Для удобства каждый элемент сетки связан со всеми своими инцидентными элементами: так связаны между собой инцидентные узлы и ребра, узлы и ячейки, ребра и ячейки. Множество инцидентных узлов будем обозначать $\mathscr{N}$, множество инцидентных ребер будем обозначать $\mathscr{E}$, а множество инцидентных ячеек будем обозначать $\mathscr{F}$.

\begin{figure}[h]
\includegraphics[width=0.48\textwidth]{pics/pic_architecture_size.pdf}
\captionstyle{center}\caption{Архитектура расчетной сетки.}\label{fig:pic_architecture}
\end{figure}

К расчетной сетке предъявляются следующие требования.
Во-первых, сетка должна быть целостной, то есть каждое ребро имеет ровно два инцидентных узла, отсутствуют изолированные и висячие узлы, а также изолированные ребра.
Во-вторых, все ячейки должны представлять собой треугольники (это гарантирует, что ячейка является плоской, так как четыре и более произвольных узлов могут не лежать в одной плоскости).
И в-третьих, рассматриваются только замкнутые сетки, представляющие собой поверхности, то есть каждое ребро имеет ровно две инцидентные ячейки.

\begin{equation}\label{eq_arch}
\begin{cases}
\forall N \Rightarrow \mathscr{E}(N) > 2, \mathscr{F}(N) > 2 \\
\forall e \Rightarrow \mathscr{N}(e) = 2 , \mathscr{F}(e) = 2 \\
\forall f \Rightarrow \mathscr{N}(f) = 3 , \mathscr{E}(f) = 3 \\
\end{cases}
\end{equation}

В качестве дополнения также будем требовать, чтобы сетка представляла собой двустороннюю поверхность, для каждой ячейки однозначно определена нормаль к поверхности $\vec{n}_f$.
Также никакие два узла сетки не совпадают и отсутствуют ячейки с нулевой площадью (так как это сделает невозможным вычислений нормалей).
Для узла сетки будем рассматривать понятие нормали к поверхности и определять эту нормаль как

\begin{equation}
\vec{n}_n(N) = \frac{1}{|\mathscr{F}(N)|} \sum_{f \in \mathscr{F}(N)}{\vec{n}_f(f)}
\end{equation}

%---------------------------------------------------------------------------------------------------

\section{Перестроение}
Центральная задача перестроения поверхности из-за нарастания ледяного покрова выглядит следующим образом.
Пусть известно, что в результате численного решения задачи ледообразования конечно-объемным методом \cite{Beaugendre} в каждой ячейке сетки была вычислена масса накопленного льда ($m$).
Будем считать плотность льда постоянной, то есть в каждой ячейке также известен объем накопленного льда ($V$).
Для каждого узла сетки $N$ требуется найти его новое положение в пространстве $N'$, чтобы для каждой ячейки с узлами $ABC$ объем пространства, ограниченный фигурой $ABCA'B'C'$ соответствовал объему льда, накопленному в данной ячейке.

Следует отметить, что поставленная задача может не иметь точного решения, и в этом случае следует стремиться к минимизации ошибки по объему (когда фактически образовавшийся объем льда не слишком сильно отличается от целевого объема, то есть разница $V_{ABCA'B'C'} - V$ мала).

Задачу определения новых положений узлов расчетной сетки можно разделить на две задачи: определение направлений смещения узлов и определение величин смещения.
Далее рассмотрим отдельные методы перестроения поверхностей более подробно.

\subsection{Классические методы перестроения}
\begin{figure}
  \centering
  \begin{minipage}[b]{0.49\textwidth}
    \includegraphics[width=\textwidth]{pics/pic_classical_methods_rectangles_size.pdf}
    \caption{Перестроение с помощью метода прямоугольников в 2D.}\label{fig:pic_classical_methods_rectangles}
  \end{minipage}
  \hfill
  \begin{minipage}[b]{0.49\textwidth}
    \includegraphics[width=\textwidth]{pics/pic_classical_methods_trapezoids_size.pdf}
    \caption{Перестроение с помощью метода трапеций в 2D.}\label{fig:pic_classical_methods_trapezoids}
  \end{minipage}
\end{figure}

\subsection{Метод Тонга}
В работах \cite{Thompson,Tong} описан устойчивый итерационный алгоритм эволюции поверхностной сетки, сохраняющий целевой объем льда.
В нем используется ряд улучшений по сравнению с классическими методами.

Многослойный подход, реализованный в этом методе, не использует константное количество шагов - величина наращиваемого объема на каждом шаге алгоритма рассчитывается исходя из максимально допустимой доли временного шага обледенения, после превышения которого возможно развитие численной нестабильности в эволюции поверхности.
Наиболее очевидный случай возникает, когда проекции нормалей граней пересекаются, в этом случае слишком большой временной шаг приведет к складыванию поверхности.
Чтобы идентифицировать грани, которые будут демонстрировать подобное поведение на текущем временном шаге, предполагается, что объем, образованный путем вытягивания треугольной грани с использованием параллельной плоскости смещения, образует призматоид, объем которого определяется кубической функцией высоты $h$:

\begin{equation}\label{Tong:1}
V(h)=ah+bh^2+ch^3
\end{equation}

где константы $a$, $b$, $c$ определяются позициями узлов грани, их нормалей и нормалью грани.
Рассмотрим корни квадратного уравнения, которое получается в результате дифференцирования уравнения \ref{Tong:1}.
Если корни являются положительными вещественными значениями, наименьший положительный корень определяет высоту, на которой достигается максимальный объем, которая обозначается как $V_{max}$, иначе функция монотонна с возрастанием и ограничение на шаг в данной грани не требуется.
Исходя из этого, можно вычислить максимальную долю временного шага обледенения, которая требуется для обеспечения разумного поведения накопления объема.
В дополнение к этому пределу размера шага был введен предел стабильности $\alpha_{jiao}$, который основан на том, как изменяются направления нормалей по мере эволюции поверхности \cite{Jiao}.
Тогда, допустимая доля временного шага для $i$-й грани определяется как

\begin{equation}\label{Tong:2}
\alpha_{\Delta t}^i=
\begin{cases}
min(s_{\Delta t}\frac{V_{max}^i}{V_f},\alpha_{jiao},1), \text{if $V_{max}^i$ exists}, \\
\alpha_{jiao}, \text{if $V_{max}^i$ doesn't exist}
\end{cases}
\end{equation}

где $s_{\Delta t}$ ($0 < s_{\Delta t} < 1$) -- эмпирически определяемый коэффициент, $V_f$ -– текущий оставшийся объем приращения льда для $i$-й грани.
Тогда объем, наращенный для текущего шага, равен $\alpha_{\Delta t} V_f$, где $\alpha_{\Delta t}$ представляет собой глобальное минимальное значение для всех граней.

Другой важной особенностью алгоритма является введение первичного и нулевого простанств, описанных в \cite{Jiao_null_space_smooth}.
Если эволюционное движение узлов сетки происходит в первичном пространстве, то их перемещение в нулевом пространстве будет сохранять потенциальную точность второго порядка триангуляции поверхности, благодаря чему мы можем  проводить сглаживание поверхности сетки с сохранением объема.
В алгоритме используется несколько видов сглаживаний.

Первое сглаживание -- сглаживание нормалей в вершинах и ячейках сетки.
Чтобы сделать возможным сглаживание в нулевом пространстве, все нормали в узлах рассчитываются так, чтобы они лежали в первичном пространстве, а перемещение узлов при наращивании льда происходит только по их нормалям.
По мере эволюции, на поверхности может усиливаться шум -- если его не контролировать, может возникнуть ситуация, когда двугранный угол между гранями станет слишком малым и ограничит максимальную долю временного шага обледенения.
Для уменьшения поверхностного шума, перед наращиванием льда применяется локальное сглаживание, регулирующее направление смещения узла в проблемных областях, чтобы оно более точно совпадало с направлениями его соседей.
Этот метод может улучшить гладкость поверхности в некоторых ситуациях.
Основная цель сглаживания нормалей -— вытолкнуть точки из вогнутых областей, где нормали могут локально сходиться.
Сглаживание нормалей достигается с помощью серий взвешенных средних, которые предназначены для придания веса нормалям, генерируемым проблемными областями.

Второе сглаживание -- сглаживание высот.
После вычисления доли временного шага и объема, наращиваемого для текущего шага, для эволюции поверхности необходимо определить поле высот, которое будет соответствовать этому объему, чтобы по нему определить смещения узлов сетки.
Решение $V(h_i) = \alpha_{\Delta t} V_f$ обеспечивает поле начальных высот, которое используется для движения поверхности.
Цель дополнительного шага сглаживания высоты состоит в том, чтобы отфильтровать высокочастотный шум в поле высот за счет уменьшения разницы высот между соседними гранями.
Как правило, высоты двух треугольных граней, имеющих общее ребро, не будут равными.
На данном шаге используется сглаживание высот с сохранением объема путем его перераспределения между соседними гранями.

\begin{figure}
  \centering
  \begin{minipage}[h]{0.49\textwidth}
    \includegraphics[width=\textwidth]{pics/pic_smooth_before.png}
    \caption{Mesh before null-space smoothing}\label{fig:pic_smooth_before}
  \end{minipage}
  \hfill
  \begin{minipage}[h]{0.49\textwidth}
    \includegraphics[width=\textwidth]{pics/pic_smooth_after.png}
    \caption{Mesh after null-space smoothing.}\label{fig:pic_smooth_after}
  \end{minipage}
\end{figure}

Последним типом сглаживания является сглаживание в нулевом пространстве.
Эволюция поверхности будет стремиться упаковать узлы в вогнутые области, где сходятся нормали к поверхности, тогда как расширение сетки происходит в выпуклых областях, где нормали к поверхности расходятся.
Если узлы не будут перераспределены, может стать невозможным продолжать продуктивный, стабильный временной шаг.
Для улучшения качества поверхностной сетки узлы перераспределяются на поверхности с помощью сглаживания в нулевом пространстве.
Этот метод способен перераспределять точки, сохраняя при этом целостность базовой геометрии.
Нулевое пространство определяется касательной плоскостью (для гладких областей), касательной линией (для складок поверхности) или пустым пространством (для углов), движущиеся в нем узлы остаются на поверхности, так что объем и форма поверхности могут быть сохранены (Fig.~\ref{fig:pic_smooth_before}, Fig.~\ref{fig:pic_smooth_after}).
\subsection{Метод общей огибающей}
Рассмотрим задачу определения новых положений узлов расчетной сетки, когда для каждого узла $\vec{N}$ известна скорость нарастания льда $v(\vec{N})$ в метрах в секунду.
Будем считать, что нарастание льда в любой точке роста выполняется одновременно во всех направлениях аналогично принципу Гюйгенса-Фреленя распространения волн.
Тогда фронт распространения льда от произвольной точки $\vec{P}$ через промежуток времени $\Delta t$ будет иметь форму сферы с центром в точке $\vec{P}$ и радиусом $v(\vec{P}) \Delta t$.
Далее будем предполагать, что выполняется расчет новых положений узлов через некоторый фиксированный момент времени $\Delta t$, то есть для каждого узла известен радиус продвижения фронта льда $R(\vec{N}) = v(\vec{N}) \Delta t$.
Так как элементами расчетной сетки являются треугольники, то необходимо определить радиус продвижения фронта льда для каждой внутренней точки треугольника по данным его вершин.

Рассмотрим ячейку расчетной сетки, вершинами которой являются точки $\vec{A}$, $\vec{B}$, $\vec{C}$.
Точки треугольника представляют собой геометрическое место точек, описываемое следующим образом:

\begin{equation}
\begin{cases}
\vec{P}(\beta, \gamma) = \vec{A} + \beta (\vec{B} - \vec{A}) + \gamma (\vec{C} - \vec{A}) = \vec{A} + \beta \vec{AB} + \gamma \vec{AC} \\
\beta \ge 0 \\
\gamma \ge 0 \\
\beta + \gamma \le 1
\end{cases}
\end{equation}

Определим для каждой точки треугольника $\vec{P}(\beta, \gamma)$ радиус продвижения фронта льда как $R(\vec{P}(\beta, \gamma)) = R(\beta, \gamma) = R(\vec{A}) + \beta (R(\vec{B}) - R(\vec{A})) + \gamma (R(\vec{C}) - R(\vec{A})) = R_A + \beta R_{AB} + \gamma R_{AC}$.
Фронт продвижения льда от точки $\vec{P}(\beta, \gamma)$ представляет собой сферу $S(\beta, \gamma) = S(\vec{P}(\beta, \gamma), R(\beta, \gamma))$.
Фронтом продвижения льда всей треугольной ячейки будем считать общую огибающую сфер, построенных на всех точках этой ячейки, что показано на Fig.~\ref{fig:pic_general_envelope_size}.

\begin{figure}
  \centering
  \begin{minipage}[b]{0.48\textwidth}
    \includegraphics[width=\textwidth]{pics/pic_general_envelope_size.pdf}
    \caption{Общая огибающая поверхность сфер, построенных на точках треугольника.}\label{fig:pic_general_envelope_size}
  \end{minipage}
  \hfill
  \begin{minipage}[b]{0.48\textwidth}
    \includegraphics[width=\textwidth]{pics/pic_general_envelope_2_size.pdf}
    \caption{Поиск нового положения узла на общей касательной плосткости трех сфер.}\label{fig:pic_general_envelope_2_size}
  \end{minipage}
\end{figure}

При изменении положения узлов расчетной сетки (точки $\vec{A}$, $\vec{B}$, $\vec{C}$) будем исходить из предположения, что после перемещения узлы будут находиться на общей огибающей поверхности множества сфер $S(\beta, \gamma)$ (новые положения узлов -- точки $\vec{A'}$, $\vec{B'}$, $\vec{C'}$).
Пока в расчете не учитываем влияние соседних расчетных ячеек.
Без ограничения общности можно рассмотреть только одну вершину ячейки (точка $\vec{A}$).
Пусть траектория движения точки $\vec{A}$ описывается уравнением полупрямой $\vec{P}(\alpha) = \vec{A} + \alpha \vec{D}$ при $\alpha \ge 0$.
$\vec{D}$ -- вектор направления движения точки, можно считать, что $|\vec{D}| = 1$.
Для поиска точек пересечения траектории движения точки $\vec{P}(\alpha) = \vec{A} + \alpha \vec{D}$ с произвольной сферой $S(\beta, \gamma)$ необходимо подставить координаты точки $\vec{P}(\alpha)$ в уравнение сферы $|\vec{P} - \vec{C}(\beta, \gamma)| = R(\beta, \gamma)$, где $\vec{C}(\beta, \gamma)$ -- центр рассматриваемой сферы.
В результате получим следующее уравнение:

\begin{equation}\label{eqn:intersect}
|(\vec{A} + \alpha \vec{D}) - \vec{C}(\beta, \gamma)| = R(\beta, \gamma)
\end{equation}

Это уравнение нужно решить относительно неизвестной $\alpha$ при фиксированных параметрах $\beta$ и $\gamma$.
Это уравнение является квадратным, оно имеет не более двух корней, которые зависят от параметров $\alpha_{1,2} = \alpha_{1,2}(\beta, \gamma)$.
Для определения нового положения точки $\vec{A}$ необходимо найти максимальное значение вещественного корня такого уравнения для всех допустимых значений параметров.
При этом точка пересечения траектории движения точки $\vec{A}$ с общей огибающей семейства сфер может находиться на разных участках этой огибающей, что продемонстрировано на Fig.~\ref{fig:pic_general_envelope_size} и связано с условиями, которым удовлетворяют параметры $\beta$ и $\gamma$ (пункты a), b), c) -- пересечение со сферой с центром в вершинах треугольника, пункты d), e), f) -- пересечение со сферой с центром на ребрах треугольника, пункт g) -- пересечение со сферой с центром внутри треугольника).

Уравнение (\ref{eqn:intersect}) можно записать в виде $|\alpha \vec{D} - (\beta \vec{AB} + \gamma \vec{AC})|^2 = (R_A + \beta R_{AB} + \gamma R_{AC})^2$ или в явном виде как квадратное уравнение:

\begin{equation}
|\vec{D}|^2 \alpha^2 - 2(\beta (\vec{D}, \vec{AB}) + \gamma (\vec{D}, \vec{AC})) \alpha + |\beta \vec{AB} + \gamma \vec{AC}|^2 - (R_A + \beta R_{AB} + \gamma R_{AC})^2 = 0
\end{equation}

Наибольший корень этого уравнения (с учетом условия $|\vec{D}| = 1$) можно выписать в явном виде:

\begin{multline}
\alpha(\beta, \gamma) = \beta (\vec{D}, \vec{AB}) + \gamma (\vec{D}, \vec{AC}) + \\
\sqrt{(\beta (\vec{D}, \vec{AB}) + \gamma (\vec{D}, \vec{AC}))^2 - |\beta \vec{AB} + \gamma \vec{AC}|^2 + (R_A + \beta R_{AB} + \gamma R_{AC})^2}
\end{multline}

или

\begin{equation}
\begin{cases}
\alpha(\beta, \gamma) = k_{\beta} \beta + k_{\gamma} \gamma + \sqrt{T} \\
T = q_{\beta^2} \beta^2 + q_{\gamma^2} \gamma^2 + q_{\beta \gamma} \beta \gamma + q_{\beta} \beta + q_{\gamma} \gamma + q \\
k_{\beta} = (\vec{D}, \vec{AB}), k_{\gamma} = (\vec{D}, \vec{AC}) \\
q_{\beta^2} = (\vec{D}, \vec{AB})^2 - |\vec{AB}|^2 + R_{AB}^2, q_{\gamma^2} = (\vec{D}, \vec{AC})^2 - |\vec{AC}|^2 + R_{AC}^2 \\
q_{\beta \gamma} = 2 ((\vec{D}, \vec{AB})(\vec{D}, \vec{AC}) - (\vec{AB}, \vec{AC}) + R_{AB} R_{AC}) \\
q_{\beta} = 2 R_A R_{AB}, q_{\gamma} = 2 R_A R_{AC}, q = R_A^2
\end{cases}
\end{equation}

Для поиска нового положения точки $\vec{A}$ требуется найти максимум выражения $\alpha(\beta, \gamma)$ при условии соблюдения ограничений $\beta \ge 0$, $\gamma \ge 0$, $\beta + \gamma \le 1$.
Максимум выражения $\alpha(\beta, \gamma)$ достигается либо при условии нахождения центра сферы внутри треугольника, либо на одной из его сторон.

В случае нахождения центра сферы на стороне $AB$ треугольника $ABC$ выполняется условие $\gamma = 0$, и выражение для величины $\alpha$ имеет вид $\alpha_{\gamma = 0}(\beta) = k_{\beta} \beta + \sqrt{q_{\beta^2} \beta^2 + q_{\beta} \beta + q}$.
В случае нахождения центра сферы на стороне $AC$ треугольника $ABC$ выполняется условие $\beta = 0$, и выражение для величины $\alpha$ имеет вид $\alpha_{\beta = 0}(\gamma) = k_{\gamma} \gamma + \sqrt{q_{\gamma^2} \gamma^2 + q_{\gamma} \gamma + q}$.
В случае нахождения центра сферы на стороне $BC$ треугольника $ABC$ выполняется условие $\beta + \gamma = 1$, и выражение для величины $\alpha$ принимает следующий вид:

\begin{multline}
\alpha_{\beta + \gamma = 1}(\gamma) = (k_{\gamma} - k_{\beta}) \gamma + k_{\beta} + \\
\sqrt{(q_{\beta^2} + q_{\gamma^2} - q_{\beta \gamma}) \gamma^2 + (-2 q_{\beta^2} + q_{\beta \gamma} - q_{\beta} + q_{\gamma}) \gamma + (q_{\beta^2} + q_{\beta} + q)}
\end{multline}

Во всех случаях нахождения центра на одной из сторон треугольника задача нахождения максимального значения $\alpha(\beta, \gamma)$ при заданных ограничениях сводится к задаче поиска максимального значения функции вида $\alpha(x) = k_x x + k + \sqrt{q_{x^2} x^2 + q_x x + q}$ (с учетом этих ограничений), что не представляет труда (точкой максимума является либо точка локального экстремума, либо точка границы области определения функции).

Отдельно следует рассмотреть вариант, при котором центр искомой сферы находится внутри треугольника. В этом случае точка пересечения траектории $\vec{P}(\alpha) = \vec{A} + \alpha \vec{D}$ с общей огибающей находится на общей касательной плоскости к сферам $S(0, 0)$, $S(1, 0)$, $S(0, 1)$ (плоскость $A'B'C'$, см. Fig.~\ref{fig:pic_general_envelope_2_size}).

Центр искомой сферы находится с точке пересечения плоскости $ABC$ и прямой, проходящей через точку $\vec{P}(\alpha)$ и направленной вдоль нормали к плоскости $A'B'C'$.
Если вектор единичной нормали плоскости $A'B'C'$ обозначить через $\vec{n}$, то взаимосвязь точек плоскостей $ABC$ и $A'B'C'$ можно выразить следующим образом:

\begin{equation}
\begin{cases}
\vec{A'} = \vec{A} + \vec{n} R_A \\
\vec{B'} = \vec{B} + \vec{n} R_B \\
\vec{C'} = \vec{C} + \vec{n} R_C
\end{cases}
\end{equation}

Для нахождения вектора $\vec{n}$ воспользуемся тем фактом, что результат векторного произведения $\vec{A'B'} \times \vec{A'C'}$ коллинеарен вектору $\vec{n}$.
Запишем это в явном виде

\begin{equation}
(\vec{AB} + \vec{n} R_{AB}) \times (\vec{AC} + \vec{n} R_{AC}) = t \vec{n}
\end{equation}

\begin{equation}
\vec{AB} \times \vec{AC} + R_{AB} (\vec{n} \times \vec{AC}) - R_{AC} (\vec{n} \times \vec{AB}) = t \vec{n}
\end{equation}

\begin{equation}
t
\left[ { \begin{array}{c}
            n_x \\
            n_y \\
            n_z \\
         \end{array} } \right]
+ R_{AC}
\left[ { \begin{array}{c}
            n_y AB_z - n_z AB_y  \\
            n_z AB_x - n_x AB_z \\
            n_x AB_y - n_y AB_x \\
         \end{array} } \right]
- R_{AB}
\left[ { \begin{array}{c}
            n_y AC_z - n_z AC_y \\
            n_z AC_x - n_x AC_z \\
            n_x AC_y - n_y AC_x \\
         \end{array} } \right]
= \vec{AB} \times \vec{AC}
\end{equation}

Данное соотношение может выполняться при $t > 0$, либо при $t < 0$, что соответствует существованию двух общих касательных плоскостей к трем сферам.
Перепишем приведенное соотношение в виде системы линейных уравнений относительно составляющих нормали при произвольном значении параметра $t$.

\begin{equation}
\left[ { \begin{array}{ccc}
             t & R_{AC} AB_z - R_{AB} AC_z & R_{AB} AC_y - R_{AC} AB_y \\
             R_{AB} AC_z - R_{AC} AB_z & t & R_{AC} AB_x - R_{AB} AC_x \\
             R_{AC} AB_y - R_{AB} AC_y & R_{AB} AC_x - R_{AC} AB_x & t \\
         \end{array} } \right]
\left[ { \begin{array}{c}
            n_x \\
            n_y \\
            n_z \\
         \end{array} } \right]
= \vec{AB} \times \vec{AC}
\end{equation}

Из данной системы уравнений находятся два возможных направления нормали к общей касательной плоскости сфер $S(0,0)$, $S(1,0)$, $S(0, 1)$.
Для каждого направления нормали находится плоскость $A'B'C'$ и точка $\vec{P}(\alpha) = \vec{A} + \alpha \vec{D}$ на ней (и сам искомый параметр $\alpha$).
Данную точку можно учитывать в общем наборе решений только в том случае, если ей соответствует сфера $S(\beta, \gamma)$, параметры которой удовлетворяют условиям $\beta \ge 0$, $\gamma \ge 0$, $\beta + \gamma \le 1$.
Параметры $\beta$ и $\gamma$ можно определить путем поиска точки пересечения прямой $\vec{P}(\alpha) - x \vec{n}$, $x \in \mathbb{R}$ и плоскости $ABC$.

Таким образом, найдя решения всех потенциально возможных частных случаев, представленных на Fig.~\ref{fig:pic_general_envelope_size}, находим множество решений $\alpha$, из которых для определения актуального нового положения точки $A'$ необходимо взять максимальное.

Изложенный выше алгоритм касается определения смещения узла с учетом только одной инцидентной ячейки.
При рассмотрении отдельного узла сетки требуется вычислить смещение данного узла относительно каждой инцидентной ячейки и выбрать среди этих смещений максимальное.

\begin{figure}
  \centering
  \begin{minipage}[b]{0.55\textwidth}
    \includegraphics[width=\textwidth]{pics/pic_general_envelope_3_size.pdf}
    \caption{TODO 1.}\label{fig:pic_general_envelope_3_size}
  \end{minipage}
  \hfill
  \begin{minipage}[b]{0.44\textwidth}
    \includegraphics[width=\textwidth]{pics/pic_general_envelope_4_size.pdf}
    \caption{TODO 2.}\label{fig:pic_general_envelope_4_size}
  \end{minipage}
\end{figure}

\begin{figure}
  \centering
  \begin{minipage}[b]{0.49\textwidth}
    \includegraphics[width=\textwidth]{pics/pic_envelope_cave.png}
    \caption{TODO 1.}\label{fig:pic_general_envelope_cave}
  \end{minipage}
  \hfill
  \begin{minipage}[b]{0.49\textwidth}
    \includegraphics[width=\textwidth]{pics/pic_envelope_peak.png}
    \caption{TODO 2.}\label{fig:pic_envelope_peak}
  \end{minipage}
\end{figure}

Алгоритм определения новых положений узлов по общей огибающей семейства сфер является линейным по количеству узлов (считаем, что на пригодных к расчетам сетках количество инцидентных ячеек для одного узла ограничено разумным числом) и не содержит итерационных процедур.
Алгоритм обладает особенностью затягивать мелкие впадины и шумы на сетке.
Это происходит из-за того, что узел $\vec{N}$, находясь на дне впадины, имеет возможность смещаться по направлению выхода из этой впадины на расстояние, большее $R(\vec{N})$, что продемонстрировано на Fig.~\ref{fig:pic_general_envelope_3_size}.
Другой интересной особенностью является работа алгоритма на участках сетки с острыми выступами (Fig.~\ref{fig:pic_general_envelope_4_size}).
Из данного рисунка видно, что в процессе работы алгоритма острые пики имеют тенденцию к сглаживанию. 
Таким образом, новые положения узлов образуют более гладкую поверхность, чем она была до перестроения.

%---------------------------------------------------------------------------------------------------

\section{Адаптация}
\subsection{Дробление ячеек}
\input{adaptation_cut}
\subsection{Стягивание по ребру}
\input{adaptation_reduce}
\subsection{Загрубление}
\input{adaptation_coarsening}

%---------------------------------------------------------------------------------------------------

\section{Устранение самопересечений}
\subsection{Поиск треугольников}
Устранение самопересечений -- поиск треугольников и точек пересечения
\subsection{Zipper}
\begin{figure}
  \centering
  \begin{minipage}[b]{0.32\textwidth}
    \includegraphics[width=\textwidth]{pics/pic_zip_01.png}
    \caption{zip 01.}\label{fig:pic_zip_01}
  \end{minipage}
  \hfill
  \begin{minipage}[b]{0.32\textwidth}
    \includegraphics[width=\textwidth]{pics/pic_zip_09.png}
    \caption{zip 09.}\label{fig:pic_zip_09}
  \end{minipage}
  \hfill
  \begin{minipage}[b]{0.32\textwidth}
    \includegraphics[width=\textwidth]{pics/pic_zip_15.png}
    \caption{zip 15.}\label{fig:pic_zip_15}
  \end{minipage}
\end{figure}
\subsection{Дробление ячеек}
\begin{figure}
  \centering
  \begin{minipage}[b]{0.49\textwidth}
    \includegraphics[width=\textwidth]{pics/pic_self_intersection_on.png}
    \caption{Поверхность до удаления самопересечения.}\label{fig:pic_self_intersection_on}
  \end{minipage}
  \hfill
  \begin{minipage}[b]{0.49\textwidth}
    \includegraphics[width=\textwidth]{pics/pic_self_intersection_off.png}
    \caption{Поверхность после удаления самопересечения.}\label{fig:pic_self_intersection_off}
  \end{minipage}
\end{figure}

\begin{figure}
  \centering
  \begin{minipage}[b]{0.49\textwidth}
    \includegraphics[width=\textwidth]{pics/pic_self_intersection_on_2.png}
    \caption{Поверхность до удаления самопересечения.}\label{fig:pic_self_intersection_on}
  \end{minipage}
  \hfill
  \begin{minipage}[b]{0.49\textwidth}
    \includegraphics[width=\textwidth]{pics/pic_self_intersection_off_2.png}
    \caption{Поверхность после удаления самопересечения.}\label{fig:pic_self_intersection_off}
  \end{minipage}
\end{figure}

%---------------------------------------------------------------------------------------------------

\section{Conclusion}
It is established the invariance of the structure of the wavelet-fractal-correlation DO type identification algorithm to the peculiarities of the current phono-target environment controlled by OED.

The statistical stability of the initial sufficient statistics for the algorithm and the statistical stability of DO type identification algorithm itself are proved for various features of its flight path and sight.

The performance indicators of the wavelet-fractal-correlation type recognition algorithm are estimated for various real-world operating conditions of the OED as the primary sensor of the sets of measurements of the coordinates of the elevation angle, azimuth and range of the detected DO at each finite current time interval and the stable performance of the algorithm in both simple and complex phono-target conditions for the functioning of the OED, including during intensive maneuvering of DO.

Set forth above it represents the development of the theory of methods for assessing the stability of the functioning of algorithms in the conditions of statistical and tactical uncertainty about the current phono-target situation in the zone of control of OED.


\begin{acknowledgements} The work was done at the Joint Supercomputer Center of the Russian Academy of Sciences - Branch of Federal State Institution Scientific Research Institute for System Analysis of the Russian Academy of Sciences within the framework of the state assignment (research topic: 065-2019-0014 (reg. no. AAAA-A19-119011590097-1)) and at the Tver State University within the framework of the state assignment (research topic: 2.1777.2017/4.6).
\end{acknowledgements}

\begin{acknowledgments}
The work was carried out at the JSCC RAS as part of the government assignment (topic FNEF-2022-0016). Supercomputer MVS-10P was used in research.
\end{acknowledgments}

%---------------------------------------------------------------------------------------------------

\begin{thebibliography}{99}

\bibitem{Beaugendre}
\refitem{misc}
H.~Beaugendre, \textquotedblleft A PDE-based approach to in-flight ice accretion,\textquotedblright \ PhD Thesis (Dep. of Mech. Eng., McGill Univ., Montr\'eal, Qu\'ebec, 2003).

\bibitem{Rybakov_2D}
\refitem{article}
A.~Rybakov and S.~Shumilin, \textquotedblleft Approximate methods of the surface mesh deformation in two-dimensional case,\textquotedblright \ Lobachevskii J Math {\bf 40}, 1848--1852 (2019).

\bibitem{BourgaultCote}
\refitem{article}
S.~Bourgault-C\^ot\'e, K.~Hasanzadeh, P.~Lavoie, and E.~Laurendeau, \textquotedblleft Multi-layer icing methodologies for conservative ice growth,\textquotedblright \ in \textit{Proceedings of 7th European Conference for Aeronautics and Aerospace Sciences EUCASS, 2017}.

\end{thebibliography}

\end{document}
